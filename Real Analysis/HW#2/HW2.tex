\documentclass[12pt]{article}

\usepackage{amssymb,amsmath,amsthm}
\usepackage[top=1in, bottom=1in, left=1.25in, right=1.25in]{geometry}
\usepackage{fancyhdr}
\usepackage{mathtools}
\usepackage{enumerate}
\usepackage{color}

%% Import a package useful for displaying graphics.
\usepackage{graphicx}

% Comment the following line to use TeX's default font of Computer Modern.
\usepackage{times,txfonts}

% Adjust line spacing an indentation
\parindent 0pt
\parskip 6pt plus 1pt

\newtheoremstyle{ex215}% name of the style to be used
  {18pt}% measure of space to leave above the theorem. E.g.: 3pt
  {12pt}% measure of space to leave below the theorem. E.g.: 3pt
  {}% name of font to use in the body of the theorem
  {}% measure of space to indent
  {\bfseries}% name of head font
  {:}% punctuation between head and body
  {2ex}% space after theorem head; " " = normal interword space
  {}% Manually specify head
\theoremstyle{ex215} 

\makeatletter
\newtheorem*{lemmacore}{Lemma \@currentlabel}
\newenvironment{lemma}[1]
{\def\@currentlabel{#1}\lemmacore}
{\endlemmacore}

\newtheorem*{corollarycore}{Corollary \@currentlabel}
\newenvironment{corollary}[1]
{\def\@currentlabel{#1}\corollarycore}
{\endcorollarycore}


%%%%%%%%%%%%%%%%%%%%%%%%%%%%%%%%%%%%%%%%%%%%%%%%%%%%%%%%%%%%%%%%%%%%%%%%
%
% Stuff for getting the name/document date/title across the header
\RequirePackage{fancyhdr}
\pagestyle{fancy}
\fancyfoot[C]{\ifnum \value{page} > 1\relax\thepage\fi}
\fancyhead[L]{\ifx\@doclabel\@empty\else\@doclabel\fi}
\fancyhead[C]{\ifx\@docauthor\@empty\else\@docdate\fi}
\fancyhead[R]{\ifx\@docauthor\@empty\@docdate\else\@docauthor\fi}
\headheight 15pt


\def\doclabel#1{\gdef\@doclabel{#1}}
\doclabel{Use {\tt\textbackslash doclabel\{MY LABEL\}}.}
\def\docdate#1{\gdef\@docdate{#1}}
\docdate{Use {\tt\textbackslash docdate\{MY DATE\}}.}
\def\docauthor#1{\gdef\@docauthor{#1}}
%\docauthor{Use {\tt\textbackslash docauthor\{MY NAME\}}.}
\let\@docauthor\@empty

\newcounter{probcount}
\newcounter{subprobcount}
\newlength\probsep
\newlength\pshrinking
\newif\iffirstprob

% The following commands allow for a handy override of labels in an enumeration,
% without having the user keep track of what level the enumeration is happening at.
% To use them, just put \listlabel{....} as the first command after a \begin{enumerate}.
% The '....' is used to define the label.  You can access the name of the current label
% counter with \thelistlabel.  Hence \listlabel{\textbf{\Alph{\thelistlabel}:}} would
% generate lists with bold labels A:, B:, C:, etc.
\newcommand{\listlabel}[1]{\def\thelistlabel{\@enumctr}%
            \expandafter\def\csname label\@enumctr\endcsname{#1}}


\newenvironment{subproblems}%
  {\begin{enumerate}\listlabel{\alph{\thelistlabel})}%
  %% Make the enumerate environment think we are making a list in a list:
  \global \@newlistfalse}%
  {\end{enumerate}}%

\newenvironment{problems}%
  {\ifhmode\unskip\par\fi\setcounter{probcount}{0}\probsep\parskip
  \sbox\@tempboxa{\textbf{9.}}\pshrinking\wd\@tempboxa\advance\pshrinking\labelsep
  \advance\linewidth -\pshrinking
  \advance\@totalleftmargin\pshrinking
  \advance\leftskip\pshrinking}%
  {\ifhmode\unskip \par\fi\advance\leftskip-\pshrinking}%

\newcommand{\problem}{%
  \setcounter{subprobcount}{0}%
  \stepcounter{probcount}%
  \def\@currentlabel{\arabic{probcount}}%
  \ifhmode
    \unskip \par
  \fi
%  \addpenalty{-4000}%
  \iffirstprob\else\addvspace\probsep\fi
  \firstprobfalse
  \hskip -\labelwidth\hskip -\labelsep 
  \hbox to\labelwidth{\hss\textbf{\arabic{probcount}.}}\hskip\labelsep
}%

%% The localhead command puts a label on top of a paragraph -- handy for labels
%% like "Solution:"
\newskip\restoreparskip
\newcommand{\localhead}[1]{\par\smallskip\textbf{#1}\nobreak\\}%
%% The following arcane code was added to make the localhead enviroment get along
%% with the ams theorem enviroment (which is based on the latex list enviroment).
%% previously, the command was {\par\smallskip\textbf{#1}\\}, but the list
%% enviroment would also add a par at the end of this paragraph, which would result
%% in an extra blank line.  Our solution now is to put in our own par, but with
%% none of the parskip glue that adds extra space between paragraphs.
%%  \restoreparskip\parskip\parskip 0pt\par\nobreak\noindent\parskip\restoreparskip}%
\newcommand\solution{\localhead{Solution:}}
\newcommand\subsol{\stepcounter{subprobcount}\localhead{Solution, part \alph{subprobcount}:}}
\newcommand\subsoleg[1]{\stepcounter{subprobcount}\par\textbf{Solution, part \alph{subprobcount}:} #1\\}
\def\solver#1{(Solution by #1)\\\vskip-12pt}


%% The default 'article' class has captions that label figures Figure 1, etc.  This is too 
%% formal for homework sets, so we change \caption so that it just puts the caption text.
%% Here I have just modified the \@makecaption command from the article class.
%% Maybe I should change this in the future to let authors decide whether or not to label a figure.

\long\def\@makecaption#1#2{%
  \vskip\abovecaptionskip
  \sbox\@tempboxa{#2}%
  \ifdim \wd\@tempboxa >\hsize
    #2\par
  \else
    \global \@minipagefalse
    \hb@xt@\hsize{\hfil\box\@tempboxa\hfil}%
  \fi
  \vskip\belowcaptionskip}

%% The default implementation of the proof environment starts a new paragraph at the start of the proof,
%% with the full parskip spacing. This only looks good following a theorem statement 
%% if \parskip is set to zero.  We've got a medium sized parskip, so we overide this 
%% questionable decision.
\renewenvironment{proof}[1][\proofname]{\par
  \pushQED{\qed}%
  \normalfont \topsep6\p@\@plus6\p@\relax
  \trivlist
  \@topsep \topsep
  \item[\hskip\labelsep
        \itshape
    #1\@addpunct{.}]\ignorespaces
}{%
  \popQED\endtrivlist\@endpefalse
}
\makeatother

% Shortcuts for blackboard bold number sets (reals, integers, etc.)
\newcommand{\Reals}{\ensuremath{\mathbb R}}
\newcommand{\Nats}{\ensuremath{\mathbb N}}
\newcommand{\Ints}{\ensuremath{\mathbb Z}}
\newcommand{\Rats}{\ensuremath{\mathbb Q}}
\newcommand{\Cplx}{\ensuremath{\mathbb C}}
\newcommand{\diam}{\text{diam}}

%% Some equivalents that some people may prefer.
\let\RR\Reals
\let\NN\Nats
\let\II\Ints
\let\CC\Cplx
\let\ZZ\Ints

\doclabel{Math F641: Homework 2}
\docauthor{Stefano Fochesatto}
\docdate{\today}
\begin{document}
\begin{problems}


% Recall how the heck we define an element in delta.
% x \in Delta if and only if x can be written as \sum_{n = 1}^{\infty} \frac{a_n}{3^n}. 
% where each a_n is either a 0 or a 2. 


\problem \textbf{Carothers 2.21} Show that any ternary decimal of the form $0.a_1a_2 \dots a_n11 \text{(base 3)}$ i.e., any finite-length decimal ending in two (or more) 1s, is not an element of $\Delta$. \\
\begin{proof} Let $x$ be a ternary decimal of the form $0.a_1a_2 \dots a_n11 \text{(base 3)}$. Recall that $\Delta$ was defined by the following recurrence relation, 
  \begin{align*}
    A_0 &= [0, 1]\\
    A_{k + 1} &= \frac{1}{3} A_k \cup \left(\frac{2}{3} + \frac{1}{3}A_k\right)\\
    \Delta &\coloneqq \bigcap A_k
  \end{align*}
  We also can describes $x$ in base 10 via, 
  \begin{equation*}
    x = \sum_{i = 1}^{n + 2} \frac{a_i}{3^i} = \sum_{i = 1}^{n} \frac{a_i}{3^i} + \frac{1}{3^{n+1}} + \frac{1}{3^{n+2}}.
  \end{equation*}
  Written in this form it is clear that $x$ lies in middle third of $A_{n+1}$, more specifically $x$ is $\frac{1}{3^{n+1}}$ greater than the right most endpoint of $\frac{1}{3}A_n$. Hence $x \not\in\Delta$.
\end{proof}
\vspace*{.15in}


\problem \textbf{Carothers 2.22} Show that $\Delta$ contains no (nonempty) open interval. In particular, show that if $x, y \in \Delta$ with $x < y$, then there is some $z \in [0, 1] \setminus \Delta$ with $x < z < y$. 
\begin{proof}Suppose that  $x, y \in \Delta$ with $x < y$. Let $x_n$ and $y_n$ be a sequence of 
  0 and 2 (the ternary decimal form) such that the following equations satisfy $x < y$,
  \begin{equation*}
    x = \sum_{i = 1}^{\infty} \frac{x_i}{3^i}, \qquad
      y = \sum_{i = 1}^{\infty} \frac{y_i}{3^i}.
  \end{equation*}
  Since $x < y$ there exists some $i$ in the both series where $x_i \neq y_i$ and $x_i = 2$ and $y_1 = 0$. This entry describes how both $x, y \in A_{i}$ but $x \in \left(\frac{2}{3} + \frac{1}{3}A_{i - 1}\right)$ and $y \in \frac{1}{3} A_{i - 1}$. Simply choose $z \in \left(\frac{1}{3} + \frac{1}{3}A_{i - 1}\right)$ and note that by definition $z \in [0, 1] \setminus \Delta$ and  $x < z < y$. 







\end{proof}
\vspace*{.15in}


% Recall the definition of sequentual continuity. 
% A function $f:A \to |RR$ is continuous at a point a if for all xn \in A such that xn \to a we know that fn \to a. 

\problem \textbf{Carothers 2.25} Define $g: \RR \to \RR$ by $g(x) = 1$ if $x \in \Delta$, and $g(x) = 0$ otherwise. At which points of $\RR$ is $g$ continuous?
\begin{proof}I assert that the function is not continuous on $\Delta$. Suppose the contrary and let $a \in \RR, \Delta$ and with $x_n \in \Delta$ such that $x_n \to a$ and by continuity $g(x_n) \to g(a)$.
  We apply the previous result to produce a new sequence of $z_n \in \Delta^c$ such that 
  $x_n < z_n < x_{n + 1}$ and clearly by sandwich theorem it follows that $z_n \to a$ however by construction we know that $f(z_n) \not\to a$, a contradiction. 



  I assert that the function is continuous on $\Delta^c$. Let $a \in \RR, \Delta^c$ and consider 
  some $x_n \to a$. By definition we can describe $\Delta^c$ as an arbitrary union of open sets and therefore 
  it is open. Thus for some $\epsilon > 0$ we can construct a $B_\epsilon(a) \subseteq \Delta^c$. Now note that since $x_n \to a$ there exists some $N$ such that for all $n \geq N$ we know that $(x_n) \subseteq B_\epsilon(a)$. Therefore by definition of $G$ it must follow that for all $n \geq N$,  $g(x_n) = 0$.


\end{proof}
\vspace*{.15in}


\problem \textbf{Carothers 2.16} The algebraic numbers are those real or complex number that are the roots of polynomials having integer coefficients. Prove that the set of algebraic numbers is countable. [Hint: First show that the set of polynomials having integer coefficients is countable.]
\begin{proof} Let $P_n$ be the set of all polynomials of degree $n$. Note that there $|P_n| = \ZZ^{n+1}$ polynomials of such degree, a countable set. Recall that a countable union of countable sets is countable, hence the set of all polynomials having integer coefficients, $P = \cup_{n = 1}^{\infty} P_n$ is countable.

  Let $p \in P_i$ and note that by the fundamental theorem of algebra $P$ has exactly $i$ roots, a countable number. We define the set of algebraic numbers, $A$ with, 
  \begin{equation*}
    A = \bigcup_{P_i \subset P} \bigcup_{p \in P_i} \{x \in \RR, \CC: p(x) = 0\}
  \end{equation*}
  Since the number of roots in $p$ is countable, the number of polynomials in $P_i$ is countable and the number of $P_i$ in $P$ is countable, $A$ is also countable. 
\end{proof}
\vspace*{.15in}


\problem \textbf{Carothers 3.7} Here is a generalization of Exercises 5 and 6. Let $f: [0, \infty) \to [0, \infty)$ be increasing and satisfy $f(0) = 0$ and $f(x) > 0$ for all $x > 0$. If $f$ also satisfies $f(x + y) \leq f(x) + f(y)$ for all $x, y \geq 0$, then $f \circ d$ is a metric whenever $d$ is a metric. Show that each of the following conditions is sufficient to endure that $f(x + y) \leq f(x) + f(y)$ for all $x, y \geq 0$:
\begin{itemize}
  \item[\textbf{(a)}] $f$ has a second derivative satisfying $f'' \leq 0$;
  

  \item[\textbf{(b)}] $f$ has a decreasing first derivative;
  \item[\textbf{(c)}] $f(x)/x$ is decreasing for $x > 0$.
% a \to b \to c then show that if c holds we have 
% b to c with mean value th
\end{itemize}

\begin{proof}  Let $f: [0, \infty) \to [0, \infty)$ be increasing and satisfy $f(0) = 0$ and $f(x) > 0$ for all $x > 0$. Suppose $f$ has a second derivative satisfying $f'' \leq 0$. Let $[a, b] \subseteq [0, \infty)$. By the mean value theorem it follows that for some $x \in (a, b)$, 
  \begin{align*}
    f''(x) &= \frac{f'(b) - f'(a)}{b - a}\\
    (b - a)f''(x) &= f'(b) - f'(a).
  \end{align*}
  Since $f''(x) \leq 0$ and $b > a$ it follows that $f'(b) - f'(a) \leq 0$ and therefore 
  $f'$ is decreasing.     
\end{proof}


\begin{proof} Let $f: [0, \infty) \to [0, \infty)$ be increasing and satisfy $f(0) = 0$ and $f(x) > 0$ for all $x > 0$. Suppose $f$ has a decreasing first derivative. 


  Let $x, y \in (0, \infty)$ such that $x \leq y$. 
  Note that for some $a \in (0, x)$ the mean value theorem applies, 
  \begin{equation*}
  f'(a) = \frac{f(x) - f(0)}{x - 0}.
  \end{equation*}
  Similarly for some $b \in (x, y)$ the mean value theorem applies, 
  \begin{equation*}
   f'(b) = \frac{f(y) - f(x)}{y - x}.
  \end{equation*}
  Since $f'$ is decreasing we know that $f'(a) \geq f'(b)$. By substitution it follows that for all $0 < x \leq y$, 
  \begin{align*}
    \frac{f(x) - f(0)}{x - 0} &\geq \frac{f(y) - f(x)}{y - x},\\
    \frac{f(x)}{x} &\geq \frac{f(y) - f(x)}{y - x},\\
    (y - x)\frac{f(x)}{x} &\geq f(y) - f(x),\\
    y\frac{f(x)}{x} - f(x) &\geq f(y) - f(x),\\
    y\frac{f(x)}{x} &\geq f(y),\\
    \frac{f(x)}{x} &\geq \frac{f(y)}{y}.
  \end{align*}
  
\end{proof}


\begin{proof} Let $f: [0, \infty) \to [0, \infty)$ be increasing and satisfy $f(0) = 0$ and $f(x) > 0$ for all $x > 0$. Suppose $f(x)/x$ is decreasing for $x > 0$.
  Let $x, y \geq 0$ and note that, 
  \begin{align*}
    \frac{f(x)}{x} &\geq \frac{f(x + y)}{x + y},\\
    f(x) &\geq f(x + y)\frac{x}{x + y}.
  \end{align*}
Similarly we find that,
\begin{align*}
  \frac{f(y)}{y} &\geq \frac{f(x + y)}{x + y},\\
  f(y) &\geq f(x + y)\frac{y}{x + y}.
\end{align*}
Summing both inequalities we get the following, 
\begin{align*}
  f(x) + f(y) &\geq f(x + y)\left(\frac{x}{x + y} + \frac{y}{x + y}\right),\\
  f(x) + f(y) &\geq f(x + y).
\end{align*}

\end{proof}
\vspace*{.15in}


\problem \textbf{Carothers 3.15} We define the \emph{diameter} of a nonempty subset $A$ of $M$ by $\diam(A) = \sup\{d(a, b);a, b \in A\}$. Show that $A$ is bounded if and only if $\diam(A)$ is finite. 
 \begin{proof} Suppose that $A$ is a bounded nonempty subset of $M$. Then there exists an $x_0 \in M$  and some $C < \infty$ such that  $d(a, x_0) \leq C$ for all $a \in A$. 
  let $a, b \in A$ and note that by the triangle inequality, 
  \begin{equation*}
    d(a, b) \leq d(a, x_0) + d(x_0, b) \leq 2C. 
  \end{equation*}  
  Therefore it follows that 
  \begin{equation*}
    \diam(A) \leq 2C < \infty. 
  \end{equation*}

  
  Now let $\diam(A)$ is finite, and for the sake of contradiction suppose $A$ is not bounded. Therefore by definition there exists an $a \in A$  such that for all $x_0 \in M$
  we know $d(a, x_0) = \infty$. 

  Clearly this is a contradiction, the supremum on the set of all distances between points can't be 
  both finite and infinite. 
\end{proof}
\vspace*{.15in}




\end{problems}
\end{document}
