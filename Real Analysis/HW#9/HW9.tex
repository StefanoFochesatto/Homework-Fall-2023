\documentclass[12pt]{article}

\usepackage{amssymb,amsmath,amsthm}
\usepackage[top=1in, bottom=1in, left=1.25in, right=1.25in]{geometry}
\usepackage{fancyhdr}
\usepackage{mathtools}
\usepackage{enumerate}
\usepackage{color}
\usepackage[english]{babel}
\newtheorem*{theorem}{Theroem 7.12}

%% Import a package useful for displaying graphics.
\usepackage{graphicx}

% Comment the following line to use TeX's default font of Computer Modern.
\usepackage{times,txfonts}

% Adjust line spacing an indentation
\parindent 0pt
\parskip 6pt plus 1pt

\newtheoremstyle{ex215}% name of the style to be used
  {18pt}% measure of space to leave above the theorem. E.g.: 3pt
  {12pt}% measure of space to leave below the theorem. E.g.: 3pt
  {}% name of font to use in the body of the theorem
  {}% measure of space to indent
  {\bfseries}% name of head font
  {:}% punctuation between head and body
  {2ex}% space after theorem head; " " = normal interword space
  {}% Manually specify head
\theoremstyle{ex215} 

\makeatletter
\newtheorem*{lemmacore}{Lemma \@currentlabel}
\newenvironment{lemma}[1]
{\def\@currentlabel{#1}\lemmacore}
{\endlemmacore}

\newtheorem*{corollarycore}{Corollary \@currentlabel}
\newenvironment{corollary}[1]
{\def\@currentlabel{#1}\corollarycore}
{\endcorollarycore}


%%%%%%%%%%%%%%%%%%%%%%%%%%%%%%%%%%%%%%%%%%%%%%%%%%%%%%%%%%%%%%%%%%%%%%%%
%
% Stuff for getting the name/document date/title across the header
\RequirePackage{fancyhdr}
\pagestyle{fancy}
\fancyfoot[C]{\ifnum \value{page} > 1\relax\thepage\fi}
\fancyhead[L]{\ifx\@doclabel\@empty\else\@doclabel\fi}
\fancyhead[C]{\ifx\@docauthor\@empty\else\@docdate\fi}
\fancyhead[R]{\ifx\@docauthor\@empty\@docdate\else\@docauthor\fi}
\headheight 15pt


\def\doclabel#1{\gdef\@doclabel{#1}}
\doclabel{Use {\tt\textbackslash doclabel\{MY LABEL\}}.}
\def\docdate#1{\gdef\@docdate{#1}}
\docdate{Use {\tt\textbackslash docdate\{MY DATE\}}.}
\def\docauthor#1{\gdef\@docauthor{#1}}
%\docauthor{Use {\tt\textbackslash docauthor\{MY NAME\}}.}
\let\@docauthor\@empty

\newcounter{probcount}
\newcounter{subprobcount}
\newlength\probsep
\newlength\pshrinking
\newif\iffirstprob

% The following commands allow for a handy override of labels in an enumeration,
% without having the user keep track of what level the enumeration is happening at.
% To use them, just put \listlabel{....} as the first command after a \begin{enumerate}.
% The '....' is used to define the label.  You can access the name of the current label
% counter with \thelistlabel.  Hence \listlabel{\textbf{\Alph{\thelistlabel}:}} would
% generate lists with bold labels A:, B:, C:, etc.
\newcommand{\listlabel}[1]{\def\thelistlabel{\@enumctr}%
            \expandafter\def\csname label\@enumctr\endcsname{#1}}


\newenvironment{subproblems}%
  {\begin{enumerate}\listlabel{\alph{\thelistlabel})}%
  %% Make the enumerate environment think we are making a list in a list:
  \global \@newlistfalse}%
  {\end{enumerate}}%

\newenvironment{problems}%
  {\ifhmode\unskip\par\fi\setcounter{probcount}{0}\probsep\parskip
  \sbox\@tempboxa{\textbf{9.}}\pshrinking\wd\@tempboxa\advance\pshrinking\labelsep
  \advance\linewidth -\pshrinking
  \advance\@totalleftmargin\pshrinking
  \advance\leftskip\pshrinking}%
  {\ifhmode\unskip \par\fi\advance\leftskip-\pshrinking}%

\newcommand{\problem}{%
  \setcounter{subprobcount}{0}%
  \stepcounter{probcount}%
  \def\@currentlabel{\arabic{probcount}}%
  \ifhmode
    \unskip \par
  \fi
%  \addpenalty{-4000}%
  \iffirstprob\else\addvspace\probsep\fi
  \firstprobfalse
  \hskip -\labelwidth\hskip -\labelsep 
  \hbox to\labelwidth{\hss\textbf{\arabic{probcount}.}}\hskip\labelsep
}%

%% The localhead command puts a label on top of a paragraph -- handy for labels
%% like "Solution:"
\newskip\restoreparskip
\newcommand{\localhead}[1]{\par\smallskip\textbf{#1}\nobreak\\}%
%% The following arcane code was added to make the localhead enviroment get along
%% with the ams theorem enviroment (which is based on the latex list enviroment).
%% previously, the command was {\par\smallskip\textbf{#1}\\}, but the list
%% enviroment would also add a par at the end of this paragraph, which would result
%% in an extra blank line.  Our solution now is to put in our own par, but with
%% none of the parskip glue that adds extra space between paragraphs.
%%  \restoreparskip\parskip\parskip 0pt\par\nobreak\noindent\parskip\restoreparskip}%
\newcommand\solution{\localhead{Solution:}}
\newcommand\subsol{\stepcounter{subprobcount}\localhead{Solution, part \alph{subprobcount}:}}
\newcommand\subsoleg[1]{\stepcounter{subprobcount}\par\textbf{Solution, part \alph{subprobcount}:} #1\\}
\def\solver#1{(Solution by #1)\\\vskip-12pt}


%% The default 'article' class has captions that label figures Figure 1, etc.  This is too 
%% formal for homework sets, so we change \caption so that it just puts the caption text.
%% Here I have just modified the \@makecaption command from the article class.
%% Maybe I should change this in the future to let authors decide whether or not to label a figure.

\long\def\@makecaption#1#2{%
  \vskip\abovecaptionskip
  \sbox\@tempboxa{#2}%
  \ifdim \wd\@tempboxa >\hsize
    #2\par
  \else
    \global \@minipagefalse
    \hb@xt@\hsize{\hfil\box\@tempboxa\hfil}%
  \fi
  \vskip\belowcaptionskip}

%% The default implementation of the proof environment starts a new paragraph at the start of the proof,
%% with the full parskip spacing. This only looks good following a theorem statement 
%% if \parskip is set to zero.  We've got a medium sized parskip, so we overide this 
%% questionable decision.
\renewenvironment{proof}[1][\proofname]{\par
  \pushQED{\qed}%
  \normalfont \topsep6\p@\@plus6\p@\relax
  \trivlist
  \@topsep \topsep
  \item[\hskip\labelsep
        \itshape
    #1\@addpunct{.}]\ignorespaces
}{%
  \popQED\endtrivlist\@endpefalse
}
\makeatother

% Shortcuts for blackboard bold number sets (reals, integers, etc.)
\newcommand{\Reals}{\ensuremath{\mathbb R}}
\newcommand{\Nats}{\ensuremath{\mathbb N}}
\newcommand{\Ints}{\ensuremath{\mathbb Z}}
\newcommand{\Rats}{\ensuremath{\mathbb Q}}
\newcommand{\Cplx}{\ensuremath{\mathbb C}}
\newcommand{\diam}[1]{\text{diam}(#1)}
\newcommand\converges[1]{\mathrel{\mathop{\longrightarrow}\limits_{#1}}}
\newcommand{\norm}[2]{\left \lVert #1 \right \rVert_{#2}}
\newcommand{\abs}[1]{\left| #1 \right|}
\newcommand{\FF}{\mathcal{F}}



\let\latexchi\chi
\makeatletter
\renewcommand\chi{\@ifnextchar_\sub@chi\latexchi}
\newcommand{\sub@chi}[2]{% #1 is _, #2 is the subscript
  \@ifnextchar^{\subsup@chi{#2}}{\latexchi^{}_{#2}}%
}
\newcommand{\subsup@chi}[3]{% #1 is the subscript, #2 is ^, #3 is the superscript
\latexchi_{#1}^{#3}%
}
\makeatother

%% Some equivalents that some people may prefer.
\let\RR\Reals
\let\NN\Nats
\let\II\Ints
\let\CC\Cplx
\let\ZZ\Ints

\doclabel{Math F641: Homework \#9}
\docauthor{Stefano Fochesatto}
\docdate{\today}
\begin{document}



\begin{problems}

  \problem Find (with proof) a function in $\mathcal{R}[a, b]$ that is not a uniform limit of step functions. 

  \begin{proof} Consider $\chi_\Delta$ and suppose for the sake of contradiction that there exists a sequence of functions $f_n \in \text{Step}[0, 1]$ such that $f_n \to \chi_\Delta$ uniformly. Note that foror all $\epsilon > 0$ there exists an $N$ such that for all $x \in [0, 1]$ it follows that, 
     \begin{equation*}
      \abs{f_N(x ) - \chi_\Delta(x )} < \epsilon.
     \end{equation*}
     Since $f_N$ is a step function with a finite step partition $\mathcal{P}$, and since $\Delta$ is uncountable, there exists an $x \in \Delta$ such that $x \in I$ where $I$ is an open interval in $\mathcal{P}$. Let $y \in I$ such that $y \neq x$. Now there are two cases which each lead to a contradiction
     
     Suppose $y \in \Delta$ then since $\Delta$ is nowhere dense there exists a $z \in I$ such that $z \not\in \Delta$. Then by uniformly continuity we get the following, 
     \begin{equation*}
      \abs{\chi_\Delta(x) - f_N(x)} < \epsilon,
     \end{equation*}
     \begin{equation*}
      \abs{ f_N(z) - \chi_\Delta(z)} < \epsilon.
     \end{equation*}
     Since $x, z \in I$ it follows that $f_N(z) = f_N(x)$ and therefore the triangle inequality it follows that, 
     \begin{equation*}
      \abs{\chi_\Delta(x) - \chi_\Delta(z)} \leq  \abs{\chi_\Delta(x) - f_N(x)} + \abs{ f_N(z) - \chi_\Delta(z)} < 2\epsilon.
     \end{equation*}
     This is clearly a contradiction as $\abs{\chi_\Delta(x) - \chi_\Delta(z)} = 1$. 

     Now suppose $y \not\in \Delta$ and note that since $x, y \in I$, the same argument to get a contradiction. 

     

  \end{proof}
  \vspace*{.15in}



  \problem Suppose $\ell: \mathcal{P}(\RR) \to [0, \infty]$. Show that $\ell$ is countably additive if and only if $\ell$ is finitely additive and countably subadditive. 
  \begin{proof}$(\rightarrow)$
    Suppose $\ell: \mathcal{P}(\RR) \to [0, \infty]$ is countably additive. Clearly $\ell$ is also finitely additive, simply consider your countable collection to be some disjoint sets $A$ and $B$ and the rest to be the emptyset. 
    
    To show that $\ell$ is countably subadditive let $\{A_i\}_{i = 1}^\infty$ such that $A_i $ are not necessarily disjoint. Construct another collection of sets, $\{B_i\}_{i = 1}^\infty$ such that $B_1 = A_1$ and  $B_i = A_i \setminus \left(\cup_{k = 1}^{i - 1}A_k\right)$ for all $i \geq 2$. We have proven that $B_k \subseteq A_k $ and $\cup_{k = 1}^n B_k = \cup_{k = 1 }^n A_k$, for all $n$. Therefore countable additivity and monotonicity it follows that, 
    \begin{equation*}
      \ell\left(\bigcup_{k = 1}^\infty A_k\right)  =  \ell\left(\bigcup_{k = 1}^\infty B_k\right) = \sum_{k = 1}^{\infty} \ell(B_k) \leq \sum_{k = 1}^{\infty} \ell(A_k). 
    \end{equation*}
  \end{proof}


  \begin{proof}$(\leftarrow)$ Suppose  $\ell: \mathcal{P}(\RR) \to [0, \infty]$ is finitely additive and countably subadditive. Now consider a disjoint collection of sets $\{A_i\}_{i = 1}^\infty$. By countable subadditivity we know, 
    \begin{equation*}
      \ell\left(\cup_{i = 1}^\infty A_i\right) \leq  \sum_{i = 1}^\infty \ell\left(A_i\right).
    \end{equation*}
    Note that for each $n$ we know that $\cup_{i = 1}^n A_i \subseteq \cup_{i = 1}^\infty A_i$ and therefore by monotonicity it follows that, 
    \begin{equation*}
      \ell\left(\cup_{i = 1}^n A_i\right) \leq \ell\left(\cup_{i = 1}^\infty A_i\right).
    \end{equation*}
    Since our sets $A_i$ are dijoint, finite additivity for all $n$ it follows that, 
    \begin{equation*}
      \sum_{i = 1}^n \ell\left(A_i\right)  = \ell\left(\cup_{i = 1}^n A_i\right)\leq \ell\left(\cup_{i = 1}^\infty A_i\right) .
    \end{equation*}
    Hence, 
    \begin{equation*}
     \sum_{i = 1}^\infty \ell\left(A_i\right) \leq \ell\left(\cup_{i = 1}^\infty A_i\right).
    \end{equation*}
    So finally we can conclude that, 
    \begin{equation*}
      \ell\left(\cup_{i = 1}^\infty A_i\right) = \sum_{i = 1}^\infty \ell\left(A_i\right).
    \end{equation*}

    
  \end{proof}
  \vspace*{.15in}






  \problem \textbf{Carothers 16.4} Given any subset $E$ of $\RR$ and any $h \in \RR$, show that $m^*(E + h) = m^*(E)$, where $E + h = \{x + h : x \in E\}$. 
  \begin{proof} Let $\{I_n\}_{n = 1}^\infty$ be a measuring cover for $E$. Note that 
    \begin{equation}
      \sum_{n = 1}^\infty m^*\left(I_n\right) = \sum_{n = 1}^\infty m^*\left(I_n + h\right).
    \end{equation}
    We want to show that $\{I_n + h\}_{n = 1}^\infty$ is a measuring cover for $E + h$ since by $(1)$ we get that $m^*(E) \leq m^*(E + h)$, and by symmetry we conclude that $m^*(E + h) = m^*(E)$. 
    
    To do that we need to show that $E + h \subseteq \cup_{n = 1}^\infty I_n + h$. Let $x \in E + h$, then by definition $x - h \in E$ and since $\{I_n\}_{n = 1}^\infty$ is a measuring cover for $E$, there exists an $I_N \in \{I_n\}_{n = 1}^\infty$ such that $x - h \in I_N$, and therefore $x \in I_N + h$. So $x \in \cup_{n = 1}^\infty I_n + h$ and thus $E + h \subseteq \cup_{n = 1}^\infty I_n + h$.
    
  \end{proof}
  \vspace*{.15in}




% One direction is free from monotonicity so $m^*(E) \leq inf\{m^*(U)  : U \text{ is open and } E \subset U \}$. 
  \problem \textbf{Carothers 16.12} Prove that $m^*(E) = \inf\{m^*(U)  : U \text{ is open and } E \subset U \}$.  
  \begin{proof} First note that by monotonicity we know immediately that, 
    \begin{equation*}
      m^*(E) \leq \inf\{m^*(U)  : U \text{ is open and } E \subset U \}.
    \end{equation*}
  
    Now we must establish, 
    \begin{equation}
      m^*(E) \geq \inf\{m^*(U)  : U \text{ is open and } E \subset U \}.
    \end{equation}
    Consider the case where $m^*(E) < \infty$ (otherwise the result is trivial) and note that to demonstrate $(2)$ we must find a sequence of open sets $U_n$ such that for a given $\epsilon > 0$ there exists an $N$ where for all $n \geq N$, 
    \begin{equation*}
      m^*(U_n) \leq m^*(E) + \epsilon. 
    \end{equation*}


    Let $\epsilon > 0$. By definition of outer measure there exists a sequence of measuring covers of $E$, sequenced by $n$, denoted $\{I_{i, n}\}_{i = 1}^\infty$ for which there exists an $N$ such that for all $n \geq N$ it follows that,
    \begin{equation*}
      \sum_{i = 1}^\infty \ell(I_{i, n}) \leq m^*(E) + \epsilon. 
    \end{equation*}
    By our definition in class of a measuring cover each $I_{i, n}$ is an open interval, and further it follows if $J_n = \cup_{i = 1}^\infty I_{i, n}$, then $J_n$ is open, $E \subseteq J_n$, and $m^*(J_n) = \sum_{i = 1}^\infty I_{i, n}$. Thus, 
    \begin{equation*}
      m^*(J_n) = \sum_{i = 1}^\infty \ell(I_{i, n}) \leq m^*(E) + \epsilon. 
    \end{equation*}  
    Having demonstrated a sequence of open sets $(J_n)$ with our desired quality $(2)$, we can conclude that $m^*(E) = \inf\{m^*(U)  : U \text{ is open and } E \subset U \}$.  
    
  \end{proof}
  \vspace*{.15in}



  \problem \textbf{Carothers 16.16} If $m^*(E) = 0$, show that $m^*(E\cup A) = m^*(A) = m^*(A \setminus E)$ for any $A$. 
  \begin{proof} Note that by monotonicity, since $A\cup E \subseteq A \subseteq A\setminus E$ we know that $m^*(A \cup E) \geq m^*(A) \geq m^*(A \setminus E)$. By countable subadditivity it follows that,
    \begin{equation*}
      m^*(A \cup E) \leq m^*(A) + m^*(E) = m^*(A). 
    \end{equation*}
    Since $A = (A \cap E)\cup (A \cap E^c)$, by countable subadditivity we also get,   
    \begin{equation*}
      m^*(A) = m^*(A \cap E) + m^*(A \cap E^c) \leq m^*(E) + m^*(A \cap E^c) = m^*(A \cap E^c).
    \end{equation*}
    Therefore $m^*(E\cup A) = m^*(A) = m^*(A \setminus E)$.

  \end{proof}
  \vspace*{.15in}




  \problem \textbf{Carothers 16.22} Let $E = \cup_{n = 1}^\infty E_n$. Show that $m^*(E) = 0$ if and only if $m^*(E_n) = 0$ for every $n$.
  \begin{proof}  Let $E = \cup_{n = 1}^\infty E_n$ and suppose $m^*(E) = 0$. Since $E_n \subseteq E$ it follows by monotonicity that $m^*(E_n) \leq m^*(E) = 0$ and thus $m^*(E_n) = 0$ for every $n$. 
  \end{proof} 
  \begin{proof}
  
  Let $E = \cup_{n = 1}^\infty E_n$ and suppose $m^*(E_n) = 0$ for every $n$. By countable subadditivity it follows that, 
  \begin{equation*}
    m^*(E) \leq \sum_{i = 1}^{\infty}m^*(E_i) = 0. 
  \end{equation*}
  Thus $m^*(E)$.

  \end{proof}
  \vspace*{.15in}


  % infinity case take, $G$ to be infinity. Now $E$ has finite measure, take a sequence of measureing sets whose outer measure converges to the infium of outermeasure, union them and intersetct them.  Countable intersection of open sets. 
  \problem \textbf{Carothers 16.24} Given a subset $E$ of $\RR$, prove that there is a $G_\delta$-set $G$, containing $E$ such that $m^*(G) = m^*(E)$. 
  \begin{proof} Let $E$ be a subset of $\RR$ and consider the case with $m^*(E)< \infty$ (otherwise let $G = \RR$ and the result follows). By Problem 4 there exists a sequence of open sets $U_n$ with $E \subset U_n$, such that $m^*(U_n) \to m^*(E)$. Now let $G = \cap_{n = 1}^\infty U_n$ and note that clearly $E \subset G$. By monotonicity $m^*(G) = \inf\{m^*(U)  : U \text{ is open and } E \subset U \}$ and by Problem 4 $m^*(G) = m^*(E)$. 
  \end{proof}
  \vspace*{.15in}  



  %proof by contradiction?
  \problem \textbf{Carothers 16.25} Suppose that $m^*(E) > 0$. Give $0 < \alpha < 1$, show that there exists an open interval $I$ such that $m^*(E \cap I) > \alpha m^*(I)$.[Hint: It is enough to consider the case $m^*(E) < \infty$. Now suppose that the conclusion fails.] 
  \begin{proof} Let $m^*(E) > 0$ and suppose there exists an $\alpha$ with $0 < \alpha < 1$ such that for every open interval $I$, $m^*(E\cap I) \leq \alpha \ell(I)$. Now let $\epsilon > 0$ and note that there exists a measuring cover $\{I_n\}_{n = 1}^\infty$ such that, 
    \begin{equation*}
      \sum_{n = 1}^\infty \ell(I_n) \leq m^*(E) + \epsilon. 
    \end{equation*}
    Now it follows that, 
    \begin{align*}
      \alpha \sum_{n = 1}^\infty \ell(I_n) &\leq \alpha\left(m^*(E) + \epsilon\right)\\ 
      \sum_{n = 1}^\infty \alpha \ell(I_n) &\leq \alpha\left(m^*(E) + \epsilon\right)\\ 
      \sum_{n = 1}^\infty \alpha m^*(E\cap I_n) &\leq \alpha\left(m^*(E) + \epsilon\right)\\
    \end{align*}
      Since $E \subseteq \cup_{n = 1}^\infty (E\cap I_n)$, by countable subadditivity it follows that, 
    \begin{equation*}
      m^*(E) \leq \alpha\left(m^*(E) + \epsilon\right).
    \end{equation*}
    Further we find that,
    \begin{equation}
      0 < m^*(E)(1 - \alpha) \leq \alpha \epsilon. 
    \end{equation}
    However clearly an $\epsilon$ can be chosen such that the $(3)$ does not hold. 
    
  \end{proof}
  \vspace*{.15in}




% proof by contradiction?
  \problem \textbf{Carothers 16.28} Fix $\alpha$ with $0 < \alpha < 1$ and repeat our "middle thirds" construction of the Cantor set except that now, at the $n$th stage, each of the $2^{n - 1}$ open intervals we discard from $[0, 1]$ is to have length $(1 - \alpha)3^{-n}$. (We still want to remove each open interval from the 'middle' of a closed interval in the current level- it is important that the closed intervals that remain turn out to be nested.) The limit of this process, a set that we will name $\Delta_\alpha$, is called the generalized Cantor set and is very much like the ordinary Cantor set. Note that $\Delta_\alpha$ is uncountable, compact, nowhere dense, and so on but has nonzero outer measure. Indeed check that $m^*(\Delta_{\alpha}) = \alpha$. (See Chapter two for an example.)[Hint: you only need upper estimates for $m^*(\Delta_\alpha)$ and $m^*({\Delta_{\alpha}}^c)$] 


  \begin{proof} By definition we know that ${\Delta_{\alpha}}^c = \cap_{n = 1}^\infty J_n$ where $J_n$ is itself the union of 'middle thirds' taken from the $n$th step in the set recurrence relation definition of ${\Delta_{\alpha}}$ . Note that each $J_n$ is a union of $2^{n - 1}$ disjoint intervals of length $(1 - \alpha)3^{-n}$, and therefore by countable subadditivity we conclude that, 
    \begin{equation*}
      m^*({\Delta_{\alpha}}^c) \leq \sum_{n = 1}^{\infty} m^*(J_n) = \sum_{n = 1}^{\infty} (1 - \alpha)2^{n - 1}3^{-n} = \dfrac{(1 - \alpha)}{3}\sum_{n = 1}^{\infty}\left(\dfrac{2}{3}\right)^{n - 1} = 1 - \alpha < 1.
    \end{equation*}
    Since $[0, 1] = {\Delta_{\alpha}}^c \cup {\Delta_{\alpha}}$ by countable subadditivity we also know that,
    \begin{equation*}
      m^*([0, 1]) \leq m^*({\Delta_{\alpha}}^c) + m^*(\Delta_\alpha)
    \end{equation*}
    Clearly since $m^*([0, 1]) = 1$ and $m^*({\Delta_{\alpha}}^c) < 1$ it must be the case that $m^*(\Delta_\alpha) > 0$ as desired. 
    
  \end{proof}
  \vspace*{.15in}




\end{problems}
\end{document}
