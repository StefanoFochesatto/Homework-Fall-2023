\documentclass[12pt]{article}

\usepackage{amssymb,amsmath,amsthm}
\usepackage[top=1in, bottom=1in, left=1.25in, right=1.25in]{geometry}
\usepackage{fancyhdr}
\usepackage{enumerate}
\usepackage{color}

%% Import a package useful for displaying graphics.
\usepackage{graphicx}

% Comment the following line to use TeX's default font of Computer Modern.
\usepackage{times,txfonts}

% Adjust line spacing an indentation
\parindent 0pt
\parskip 6pt plus 1pt

\newtheoremstyle{ex215}% name of the style to be used
  {18pt}% measure of space to leave above the theorem. E.g.: 3pt
  {12pt}% measure of space to leave below the theorem. E.g.: 3pt
  {}% name of font to use in the body of the theorem
  {}% measure of space to indent
  {\bfseries}% name of head font
  {:}% punctuation between head and body
  {2ex}% space after theorem head; " " = normal interword space
  {}% Manually specify head
\theoremstyle{ex215} 

\makeatletter
\newtheorem*{lemmacore}{Lemma \@currentlabel}
\newenvironment{lemma}[1]
{\def\@currentlabel{#1}\lemmacore}
{\endlemmacore}

\newtheorem*{corollarycore}{Corollary \@currentlabel}
\newenvironment{corollary}[1]
{\def\@currentlabel{#1}\corollarycore}
{\endcorollarycore}


%%%%%%%%%%%%%%%%%%%%%%%%%%%%%%%%%%%%%%%%%%%%%%%%%%%%%%%%%%%%%%%%%%%%%%%%
%
% Stuff for getting the name/document date/title across the header
\RequirePackage{fancyhdr}
\pagestyle{fancy}
\fancyfoot[C]{\ifnum \value{page} > 1\relax\thepage\fi}
\fancyhead[L]{\ifx\@doclabel\@empty\else\@doclabel\fi}
\fancyhead[C]{\ifx\@docauthor\@empty\else\@docdate\fi}
\fancyhead[R]{\ifx\@docauthor\@empty\@docdate\else\@docauthor\fi}
\headheight 15pt


\def\doclabel#1{\gdef\@doclabel{#1}}
\doclabel{Use {\tt\textbackslash doclabel\{MY LABEL\}}.}
\def\docdate#1{\gdef\@docdate{#1}}
\docdate{Use {\tt\textbackslash docdate\{MY DATE\}}.}
\def\docauthor#1{\gdef\@docauthor{#1}}
%\docauthor{Use {\tt\textbackslash docauthor\{MY NAME\}}.}
\let\@docauthor\@empty

\newcounter{probcount}
\newcounter{subprobcount}
\newlength\probsep
\newlength\pshrinking
\newif\iffirstprob

% The following commands allow for a handy override of labels in an enumeration,
% without having the user keep track of what level the enumeration is happening at.
% To use them, just put \listlabel{....} as the first command after a \begin{enumerate}.
% The '....' is used to define the label.  You can access the name of the current label
% counter with \thelistlabel.  Hence \listlabel{\textbf{\Alph{\thelistlabel}:}} would
% generate lists with bold labels A:, B:, C:, etc.
\newcommand{\listlabel}[1]{\def\thelistlabel{\@enumctr}%
            \expandafter\def\csname label\@enumctr\endcsname{#1}}


\newenvironment{subproblems}%
  {\begin{enumerate}\listlabel{\alph{\thelistlabel})}%
  %% Make the enumerate environment think we are making a list in a list:
  \global \@newlistfalse}%
  {\end{enumerate}}%

\newenvironment{problems}%
  {\ifhmode\unskip\par\fi\setcounter{probcount}{0}\probsep\parskip
  \sbox\@tempboxa{\textbf{9.}}\pshrinking\wd\@tempboxa\advance\pshrinking\labelsep
  \advance\linewidth -\pshrinking
  \advance\@totalleftmargin\pshrinking
  \advance\leftskip\pshrinking}%
  {\ifhmode\unskip \par\fi\advance\leftskip-\pshrinking}%

\newcommand{\problem}{%
  \setcounter{subprobcount}{0}%
  \stepcounter{probcount}%
  \def\@currentlabel{\arabic{probcount}}%
  \ifhmode
    \unskip \par
  \fi
%  \addpenalty{-4000}%
  \iffirstprob\else\addvspace\probsep\fi
  \firstprobfalse
  \hskip -\labelwidth\hskip -\labelsep 
  \hbox to\labelwidth{\hss\textbf{\arabic{probcount}.}}\hskip\labelsep
}%

%% The localhead command puts a label on top of a paragraph -- handy for labels
%% like "Solution:"
\newskip\restoreparskip
\newcommand{\localhead}[1]{\par\smallskip\textbf{#1}\nobreak\\}%
%% The following arcane code was added to make the localhead enviroment get along
%% with the ams theorem enviroment (which is based on the latex list enviroment).
%% previously, the command was {\par\smallskip\textbf{#1}\\}, but the list
%% enviroment would also add a par at the end of this paragraph, which would result
%% in an extra blank line.  Our solution now is to put in our own par, but with
%% none of the parskip glue that adds extra space between paragraphs.
%%  \restoreparskip\parskip\parskip 0pt\par\nobreak\noindent\parskip\restoreparskip}%
\newcommand\solution{\localhead{Solution:}}
\newcommand\subsol{\stepcounter{subprobcount}\localhead{Solution, part \alph{subprobcount}:}}
\newcommand\subsoleg[1]{\stepcounter{subprobcount}\par\textbf{Solution, part \alph{subprobcount}:} #1\\}
\def\solver#1{(Solution by #1)\\\vskip-12pt}


%% The default 'article' class has captions that label figures Figure 1, etc.  This is too 
%% formal for homework sets, so we change \caption so that it just puts the caption text.
%% Here I have just modified the \@makecaption command from the article class.
%% Maybe I should change this in the future to let authors decide whether or not to label a figure.

\long\def\@makecaption#1#2{%
  \vskip\abovecaptionskip
  \sbox\@tempboxa{#2}%
  \ifdim \wd\@tempboxa >\hsize
    #2\par
  \else
    \global \@minipagefalse
    \hb@xt@\hsize{\hfil\box\@tempboxa\hfil}%
  \fi
  \vskip\belowcaptionskip}

%% The default implementation of the proof environment starts a new paragraph at the start of the proof,
%% with the full parskip spacing. This only looks good following a theorem statement 
%% if \parskip is set to zero.  We've got a medium sized parskip, so we overide this 
%% questionable decision.
\renewenvironment{proof}[1][\proofname]{\par
  \pushQED{\qed}%
  \normalfont \topsep6\p@\@plus6\p@\relax
  \trivlist
  \@topsep \topsep
  \item[\hskip\labelsep
        \itshape
    #1\@addpunct{.}]\ignorespaces
}{%
  \popQED\endtrivlist\@endpefalse
}
\makeatother

% Shortcuts for blackboard bold number sets (reals, integers, etc.)
\newcommand{\Reals}{\ensuremath{\mathbb R}}
\newcommand{\Nats}{\ensuremath{\mathbb N}}
\newcommand{\Ints}{\ensuremath{\mathbb Z}}
\newcommand{\Rats}{\ensuremath{\mathbb Q}}
\newcommand{\Cplx}{\ensuremath{\mathbb C}}
%% Some equivalents that some people may prefer.
\let\RR\Reals
\let\NN\Nats
\let\II\Ints
\let\CC\Cplx

\doclabel{Math F641: Homework 1}
\docauthor{Stefano Fochesatto}
\docdate{\today}
\begin{document}
\begin{problems}

\problem \textbf{Carothers 1.4} Let $A$ be a nonempty subset of $\RR$ that is bounded above. Show that there is a sequence $x_n$ of elements of $A$ that converge to $\sup A$.
\begin{proof} Suppose $A$ is a nonempty subset of $\RR$ that is bounded above. Let $s = \sup A$ and consider the sequence $x_n = s - \frac{1}{n}$.

  Note that $x_n\to s$, let $\epsilon > 0$ and choose $N \in \NN$ such that $1/N < \epsilon$ and note that for all $n \geq N$,
  \begin{align*}
    |x_n - s| &= |s - \frac{1}{n} -s|\\
      &= \frac{1}{n} \leq \frac{1}{N} < \epsilon.
  \end{align*}

  Since $s$ is the supremum of $A$ we know that for all $x_n < s$ there exists some $a_n \in A$ such that $x_n < a_n \leq s$. 
  Hence the sequence $a_n \to S$.  
\end{proof}
\vspace*{.15in}



\problem \textbf{Carothers 1.11} Fix $a>0$ and let $x_1 >\sqrt{a}$. For $n\geq 1$, define, 
\begin{equation*}
  x_{n+1} = \frac{1}{2} \left(x_n + \frac{a }{x_n }\right)
\end{equation*}
Show that $(x_n)$ converges and that $\lim_{n \to \infty }x_n = \sqrt{a }$.
\begin{proof} Suppose $a>0$ and let $x_1 >\sqrt{a}$. Since $x_1 > \sqrt{a} > 0$ it follows that $x_n > 0$. We will proceed to show that $x_n$ is bounded below by $\sqrt{a}$. Note that,
  \begin{align*}
    x_{n+1} &= \frac{1}{2} \left(x_n + \frac{a }{x_n }\right),\\
    2x_{n+1} &= x_n + \frac{a }{x_n },\\
    -x_n + 2x_{n+1}&=\frac{a }{x_n },\\
    -x_n^2 + 2x_{n+1}x_n&=a,\\
    -x_n^2 + 2x_{n+1}x_n - a&=0,\\
    x_n^2 - 2x_{n+1}x_n + a=0.
  \end{align*}
  Arriving at an equation that is quadratic with respect to $x_n$ it follows that it's discriminant, 
  $(-2x_{n+1})^2 - 4a = 4x_{n+1}^2 - 4a \geq 0$ which implies that $x_{n+1} \geq \sqrt{a}$ and therefore $x_n \geq \sqrt{a}$.

  Now we will demonstrate that $x_{n+1} \leq x_n$. From the previous result we find that,
  \begin{align*}
    \sqrt{a} &\leq x_n,\\
    a &\leq x_n^2,\\
    \frac{a}{x_n} &\leq x_n,\\
    \frac{a}{2x_n} &\leq \frac{x_n}{2},\\
    \frac{x_n}{2} + \frac{a}{2x_n} &\leq x_n,\\
    \frac{1}{2}\left(x_n + \frac{a}{x_n}\right) &\leq x_n,\\
    \frac{1}{2}\left(x_n + \frac{a}{x_n}\right) &\leq x_n,\\
    x_{n+1} &\leq x_n.
  \end{align*}
  
  Thus $x_n$ is a monotone decreasing bounded sequence and therefore converges to some limit $\lim_{n \to \infty}x_n = L$. By substitution we find that, 
  \begin{align*}
    \lim_{n \to \infty }x_{n+1}  &= \lim_{n \to \infty }\frac{1}{2} \left(x_n + \frac{a }{x_n }\right).\\
  \end{align*}
  Note that $L \neq 0$ since that would force the $\frac{a}{x_n}$ to infinity, contradiction $x_n \geq x_{n + 1}$. Therefore, 
  \begin{align*}
    L  &= \frac{1}{2} \left(L + \frac{a }{L}\right),\\
    L  &=  \frac{a }{L},\\
    L^2  &=  a,\\
    L  &=  \sqrt{a}.
  \end{align*}
\end{proof}
\vspace*{.15in}

\problem \textbf{Carothers 1.15} Show that a Cauchy sequence with a convergent subsequence actually converges.

\begin{proof} Suppose $(x_n)$ is a Cauchy sequence and $(x_n)_i \to a$ is a convergent subsequence.
  Let $\epsilon > 0$. Since $(x_n)_i \to a$ there exists an $I \in \NN$ such that for all $i \geq I$ it follows that, 
  \begin{equation*}
    |x_{n_i} - a| < \frac{\epsilon}{2}. 
  \end{equation*}

  Similarly since $(x_n)$ is Cauchy we know that there exists an $\hat{N} \geq 1$ such that 
  when $n, m \geq \hat{N}$ it follows that, 
  \begin{equation*}
    |x_{n} - x_{m}| < \frac{\epsilon}{2}. 
  \end{equation*}

  Let $\epsilon > 0$, and note that for $N = \max\{n_I, \hat{N}\}$ all $n \geq N$ have the property that, 
  \begin{align*}
    |x_{n} - a| &= |x_{n} - x_{n_{N}} + x_{n_{N}} - a|,\\
    &\leq |x_{n} - x_{n_{N}}| + |x_{n_{N}} - a|,\\
  \end{align*}
  Note that we use $x_{n_{N}}$ to ensure that $n_{N} \geq \hat{N}, n_I$. So we conclude with, 
  \begin{equation*}
    |x_{n} - a| < \frac{\epsilon}{2} + \frac{\epsilon}{2} = \epsilon.
  \end{equation*}

\end{proof}







\problem \textbf{Carothers 1.21} Let $p \geq 2$ be a fixed integer, and let $0 < x < 1$. If $x$ has a finite-length base $p$ decimal expansion, that is, if $x = a_1/p + \dots a_n/p^n$ with $a_n \neq 0$, prove that $x$ has precisely two base $p$ decimal expansions. Otherwise, show that the base $p$ decimal expansion for $x$ is unique. Characterize the numbers $0 < x <1$ that have repeating base $p$ decimal expansions. How about eventually repeating?


\problem \textbf{Carothers 1.24} Show that $\lim \sup_{n \to \infty} (-a_n) = - \lim \inf_{n \to \infty} a_n$. 
\begin{proof} Let $(a_n)$ be a bounded sequence of real numbers. Note that by definition, 
  \begin{equation*}
    \lim \sup_{n \to \infty} (-a_n) = \overline{\lim_{n \to \infty}} -a_n = \inf_{n \geq 1}\left(\sup\{-a_n, -a_{n+1}, -a_{n+2},\dots\}\right)
  \end{equation*}
  Let $-a_i$ be an eventual upper bound for the sequence and note that for all $n \geq i$ we know that $-a_i \geq -a_n$. Multiplying both sides by -1, we reverse the inequality to get $a_i \leq a_n$ and conclude that $a_i$ is an eventual lower bound for the sequence $a_n$. Hence it follows that 
  \begin{equation*}
    \inf_{n \geq 1}\left(\sup\{-a_n, -a_{n+1}, -a_{n+2},\dots\}\right) = -\sup_{n \geq 1}\left(\inf\{a_n, a_{n+1}, a_{n+2},\dots\}\right)
  \end{equation*}
  
\end{proof}



\problem Suppose $\limsup_{n\rightarrow\infty} x_n = -\infty$, as defined in terms
of eventual upper bounds.  Show that 
$$
\overline{\lim_{n\rightarrow\infty}} x_n =-\infty,
$$ as defined in the text.
\begin{proof} Suppose that $\limsup_{n\rightarrow\infty} x_n = -\infty$. Defined in terms of eventual upper bounds, 
  \begin{equation*}
    \limsup_{n\rightarrow\infty}^* x_n = \inf\{M: M \text{ is an eventual upper bound for } (x_n)\} = -\infty
  \end{equation*}
  Recall the definition of $\limsup_{n\rightarrow\infty} x_n$ from the text,
  \begin{equation*}
    \limsup_{n\rightarrow\infty} x_n = \inf_{n \geq 1}\left(\sup\{a_n, a_{n+1}, a_{n+2},\dots\}\right)
  \end{equation*}
  Note that $\sup\{a_n, a_{n+1}, ...\}$ describes an eventual upper bound with index $M \geq n$ which is also an element in the sequence $(a_n)$. Therefore the text's definition $\limsup_{n\rightarrow\infty} x_n $ describes an infimum over a subset of eventual upper bounds.

  Suppose for the sake of contradiction that $\limsup_{n\rightarrow\infty} x_n \geq L$ with $L \in \RR$. By $\limsup_{n\rightarrow\infty}^* x_n = -\infty$ we know there exists 
  an eventual upper bound $L^*$ such that $L^* < L$. If $L^* \in (x_n)$ then it follows that 
  $\limsup_{n\rightarrow\infty} x_n  = L^*$ a contradiction. Otherwise if $L^* \not\in (x_n)$
  it follows from the definition of E.U.B that for some $M \in \NN$ all $n \geq M$, $x_n < L$. 
  Finally we arrive at a contradiction
  \begin{equation*}
    \sup\{a_M, a_{M+1}, a_{n+2},\dots\}<L^* < L.
  \end{equation*}
  Therefore $\limsup_{n\rightarrow\infty} x_n  = -\infty$.
  
  
\end{proof}








\problem Let $(r_n)$ be an enumeration of $\mathbb Q\cap [0,1]$.  Show that
$\limsup{n\rightarrow\infty} = 1$.
\begin{proof} Suppose $(r_n)$ be an enumeration of $\mathbb Q\cap [0,1]$. Clearly 1 is an upper bound for  $(r_n)$ and therefore an eventual least upper bound. We will proceed to show that 1 is an infimum on the set of eventual upper bounds, and therefore $\limsup{n\rightarrow\infty} = 1$.

  Suppose for the sake of contradiction that an eventual upper bound $M$ such that $M < 1$. Therefore there exists some $N$ where for all $n \geq N$ we know that $r_n \leq M$. However by the density 
  of $\mathbb Q$ in $\RR$, there exists $N+1$ rational numbers in $(M,1)$ and hence there must exists some $r_i > M$ where $i > N$, a contradiction.
\end{proof}
%\hproblem Carothers 1.25

\problem Prove that
$$
\limsup x_n + \liminf y_n \le \limsup (x_n+y_n) \le \limsup x_n + \limsup y_n
$$
so long as neither of the right- or left-hand sides are of the form $\infty-\infty$.









\problem \textbf{Carothers 1.36} Let $a_n \geq 0$.
\begin{enumerate}
  \item[\textbf{(i)}] If $\limsup_{n\rightarrow\infty} \sqrt[n]{a_n} < 1$, show that $\sum_{n = 1}^{\infty} a_n < \infty$. 
  \begin{proof} Suppose $\limsup_{n\rightarrow\infty} \sqrt[n]{a_n} < 1$ and let $m = \limsup_{n\rightarrow\infty} \sqrt[n]{a_n}$. Note that there exists an eventual upper bound to the sequence, $M$
    such that $m< M < 1$. Therefore for some $N \in \NN$, for $n \geq N$ we know that $\sqrt[n]{a_n} < M$.
    Therefore by substitution we get, 
    \begin{equation*}
      \sum_{n = 1}^{\infty} a_n   =  \sum_{n = 1}^{N} a_n + \sum_{n = N+1}^{\infty} a_n < \sum_{n = 1}^{N} a_n + \sum_{n = N+1}^{\infty} (M)^n.
    \end{equation*} 
    Clearly $\sum_{n = N+1}^{\infty} (M)^n < \infty$ since it is a convergent geometric series, and $\sum_{n = 1}^{N} a_n < \infty$ since it is a finite sum. Therefore $\sum_{n = 1}^{\infty} a_n < \infty$.
  \end{proof} 


  \item[\textbf{(ii)}] If $\liminf_{n\rightarrow\infty} \sqrt[n]{a_n} >1$, show that $\sum_{n = 1}^{\infty} a_n$ diverges. 
  \begin{proof} Suppose $M = \liminf_{n\rightarrow\infty} \sqrt[n]{a_n} >1$ and for the sake of contradiction 
    we suppose $\sum_{n = 1}^{\infty} a_n$ converges. For the infinite series to converge it is necessary for $\lim_{n \to \infty}(a_n) \to 0$ and therefore $\liminf_{n\rightarrow\infty} {a_n} = 0$.
    Note that there exists some eventual lower bound $m$ for the sequence $\sqrt[n]{a_n}$
    such that $1<m<M$. Therefore for some $N \in \NN$, for $n \geq N$ we know that $\sqrt[n]{a_n} > 1$ and thus $a_n > 1$. To conclude, we have shown that there exists a tail of the sequence $(a_n)$ which is above 1, yet still convergent to 0, a contradiction. 
  \end{proof}



  \item[\textbf{(iii)}] Find examples of both a convergent and divergent series having $\lim_{n\to\infty} \sqrt[n]{a_n} = 1$. 







\end{enumerate}

\end{problems}
\end{document}
