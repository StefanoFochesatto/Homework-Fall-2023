\documentclass[12pt]{article}

\usepackage{amssymb,amsmath,amsthm}
\usepackage[top=1in, bottom=1in, left=1.25in, right=1.25in]{geometry}
\usepackage{fancyhdr}
\usepackage{mathtools}
\usepackage{enumerate}
\usepackage{color}
\usepackage[english]{babel}
\newtheorem*{theorem}{Theroem 7.12}

%% Import a package useful for displaying graphics.
\usepackage{graphicx}

% Comment the following line to use TeX's default font of Computer Modern.
\usepackage{times,txfonts}

% Adjust line spacing an indentation
\parindent 0pt
\parskip 6pt plus 1pt

\newtheoremstyle{ex215}% name of the style to be used
  {18pt}% measure of space to leave above the theorem. E.g.: 3pt
  {12pt}% measure of space to leave below the theorem. E.g.: 3pt
  {}% name of font to use in the body of the theorem
  {}% measure of space to indent
  {\bfseries}% name of head font
  {:}% punctuation between head and body
  {2ex}% space after theorem head; " " = normal interword space
  {}% Manually specify head
\theoremstyle{ex215} 

\makeatletter
\newtheorem*{lemmacore}{Lemma \@currentlabel}
\newenvironment{lemma}[1]
{\def\@currentlabel{#1}\lemmacore}
{\endlemmacore}

\newtheorem*{corollarycore}{Corollary \@currentlabel}
\newenvironment{corollary}[1]
{\def\@currentlabel{#1}\corollarycore}
{\endcorollarycore}


%%%%%%%%%%%%%%%%%%%%%%%%%%%%%%%%%%%%%%%%%%%%%%%%%%%%%%%%%%%%%%%%%%%%%%%%
%
% Stuff for getting the name/document date/title across the header
\RequirePackage{fancyhdr}
\pagestyle{fancy}
\fancyfoot[C]{\ifnum \value{page} > 1\relax\thepage\fi}
\fancyhead[L]{\ifx\@doclabel\@empty\else\@doclabel\fi}
\fancyhead[C]{\ifx\@docauthor\@empty\else\@docdate\fi}
\fancyhead[R]{\ifx\@docauthor\@empty\@docdate\else\@docauthor\fi}
\headheight 15pt


\def\doclabel#1{\gdef\@doclabel{#1}}
\doclabel{Use {\tt\textbackslash doclabel\{MY LABEL\}}.}
\def\docdate#1{\gdef\@docdate{#1}}
\docdate{Use {\tt\textbackslash docdate\{MY DATE\}}.}
\def\docauthor#1{\gdef\@docauthor{#1}}
%\docauthor{Use {\tt\textbackslash docauthor\{MY NAME\}}.}
\let\@docauthor\@empty

\newcounter{probcount}
\newcounter{subprobcount}
\newlength\probsep
\newlength\pshrinking
\newif\iffirstprob

% The following commands allow for a handy override of labels in an enumeration,
% without having the user keep track of what level the enumeration is happening at.
% To use them, just put \listlabel{....} as the first command after a \begin{enumerate}.
% The '....' is used to define the label.  You can access the name of the current label
% counter with \thelistlabel.  Hence \listlabel{\textbf{\Alph{\thelistlabel}:}} would
% generate lists with bold labels A:, B:, C:, etc.
\newcommand{\listlabel}[1]{\def\thelistlabel{\@enumctr}%
            \expandafter\def\csname label\@enumctr\endcsname{#1}}


\newenvironment{subproblems}%
  {\begin{enumerate}\listlabel{\alph{\thelistlabel})}%
  %% Make the enumerate environment think we are making a list in a list:
  \global \@newlistfalse}%
  {\end{enumerate}}%

\newenvironment{problems}%
  {\ifhmode\unskip\par\fi\setcounter{probcount}{0}\probsep\parskip
  \sbox\@tempboxa{\textbf{9.}}\pshrinking\wd\@tempboxa\advance\pshrinking\labelsep
  \advance\linewidth -\pshrinking
  \advance\@totalleftmargin\pshrinking
  \advance\leftskip\pshrinking}%
  {\ifhmode\unskip \par\fi\advance\leftskip-\pshrinking}%

\newcommand{\problem}{%
  \setcounter{subprobcount}{0}%
  \stepcounter{probcount}%
  \def\@currentlabel{\arabic{probcount}}%
  \ifhmode
    \unskip \par
  \fi
%  \addpenalty{-4000}%
  \iffirstprob\else\addvspace\probsep\fi
  \firstprobfalse
  \hskip -\labelwidth\hskip -\labelsep 
  \hbox to\labelwidth{\hss\textbf{\arabic{probcount}.}}\hskip\labelsep
}%

%% The localhead command puts a label on top of a paragraph -- handy for labels
%% like "Solution:"
\newskip\restoreparskip
\newcommand{\localhead}[1]{\par\smallskip\textbf{#1}\nobreak\\}%
%% The following arcane code was added to make the localhead enviroment get along
%% with the ams theorem enviroment (which is based on the latex list enviroment).
%% previously, the command was {\par\smallskip\textbf{#1}\\}, but the list
%% enviroment would also add a par at the end of this paragraph, which would result
%% in an extra blank line.  Our solution now is to put in our own par, but with
%% none of the parskip glue that adds extra space between paragraphs.
%%  \restoreparskip\parskip\parskip 0pt\par\nobreak\noindent\parskip\restoreparskip}%
\newcommand\solution{\localhead{Solution:}}
\newcommand\subsol{\stepcounter{subprobcount}\localhead{Solution, part \alph{subprobcount}:}}
\newcommand\subsoleg[1]{\stepcounter{subprobcount}\par\textbf{Solution, part \alph{subprobcount}:} #1\\}
\def\solver#1{(Solution by #1)\\\vskip-12pt}


%% The default 'article' class has captions that label figures Figure 1, etc.  This is too 
%% formal for homework sets, so we change \caption so that it just puts the caption text.
%% Here I have just modified the \@makecaption command from the article class.
%% Maybe I should change this in the future to let authors decide whether or not to label a figure.

\long\def\@makecaption#1#2{%
  \vskip\abovecaptionskip
  \sbox\@tempboxa{#2}%
  \ifdim \wd\@tempboxa >\hsize
    #2\par
  \else
    \global \@minipagefalse
    \hb@xt@\hsize{\hfil\box\@tempboxa\hfil}%
  \fi
  \vskip\belowcaptionskip}

%% The default implementation of the proof environment starts a new paragraph at the start of the proof,
%% with the full parskip spacing. This only looks good following a theorem statement 
%% if \parskip is set to zero.  We've got a medium sized parskip, so we overide this 
%% questionable decision.
\renewenvironment{proof}[1][\proofname]{\par
  \pushQED{\qed}%
  \normalfont \topsep6\p@\@plus6\p@\relax
  \trivlist
  \@topsep \topsep
  \item[\hskip\labelsep
        \itshape
    #1\@addpunct{.}]\ignorespaces
}{%
  \popQED\endtrivlist\@endpefalse
}
\makeatother

% Shortcuts for blackboard bold number sets (reals, integers, etc.)
\newcommand{\Reals}{\ensuremath{\mathbb R}}
\newcommand{\Nats}{\ensuremath{\mathbb N}}
\newcommand{\Ints}{\ensuremath{\mathbb Z}}
\newcommand{\Rats}{\ensuremath{\mathbb Q}}
\newcommand{\Cplx}{\ensuremath{\mathbb C}}
\newcommand{\diam}[1]{\text{diam}(#1)}
\newcommand\converges[1]{\mathrel{\mathop{\longrightarrow}\limits_{#1}}}
\newcommand{\norm}[1]{\left\lVert#1\right\rVert}

%% Some equivalents that some people may prefer.
\let\RR\Reals
\let\NN\Nats
\let\II\Ints
\let\CC\Cplx
\let\ZZ\Ints

\doclabel{Math F641: Homework 5}
\docauthor{Stefano Fochesatto}
\docdate{\today}
\begin{document}
\begin{problems}

I WANT USE MY ONE TIME, THIS HOMEWORK IS MOSTLY DONE BUT NOT FINISHED
%_________________________________________________________________________
%CHAPTER 7

\problem \textbf{Carothers 7.15} Prove or disprove: if $M$ is complete and $f: (M, d) \to (N \rho)$ is continuous, then $f(M)$ is complete. 
\begin{proof} Consider $\RR$ under the usual metric, and let $f$ be the $tan^{-1}$ function, so therefore $f(\RR) = (-\pi/2, \pi/2)$ which is certainly not complete since $\pi/2 - 1/n$ is a Cauchy sequence which does not converge in $f(\RR)$
\end{proof}
\vspace*{.15in}



\problem \textbf{Carothers 7.18} (Note Done) Fill in the details of the proofs that $\ell_1$ and $\ell_\infty$ are complete. 

%\begin{proof} Let $(f_n)\subseteq \ell_1$ and suppose that $(f_n)$ is Cauchy in $\ell_1$. First we propose a candidate limit with $f(k) = \lim_{n \to \infty}f_n(k)$. Of course we must first show that this limit exists, for each $k$. Since $f_n$ is Cauchy note there exists and $N$ such that if $n, m \geq N$ then, 
%  \begin{equation*}
%    ||f_n - f_m||_1 < \epsilon. 
%  \end{equation*} 
%  Note that for a fixed $k$ it follows that, 
%  \begin{equation*}
%    |f_n(k) - f_m(k)| \leq ||f_n - f_m||_1 < \epsilon. 
%  \end{equation*} 
%  Therefore the sequence $(f_n(k))$ is Cauchy in $\RR$, and since $\RR$ is complete $(f_n(k))$ converges so $f$ exists. \\
%
%  Now we show that $f \in \ell_1$. We have shown that Cauchy sequences are bounded, and therefore $||(f_n)||_1 \leq B$ for all $n$. Therefore it follows that, 
%  \begin{equation*}
%    ||f||_1 = \lim_{n \to \infty} ||f_n||_1 \leq B.
%  \end{equation*}
%  So therefore $||f||_1 \leq B$ and we conclude that by definition $f \in \ell_1$. 
%
%\end{proof}
\vspace*{.15in}



\problem \textbf{Carothers 7.19} Prove that $c_0$ is complete by showing that $c_0$ is closed in $\ell_{\infty}$.
\begin{proof} Let $(x_n) \subseteq c_0$ and $x_n \to x$ with $x \in \ell_\infty$. For $\epsilon > 0$ there exists an $N$ such that,
  \begin{equation*}
    ||x_N - x||_{\infty} = \sup_k \{ |x_N(k) - x(k)|\} < \epsilon/2.
  \end{equation*}
  %So we are going far enough out in the sequence $(x_n)$ such that this statement about %the boundedness of the difference of each term is true for all $k$. 

  Note that since $x_N \in c_0$ we know that $x_N(k) \to 0$, so there exists a $K$ such that if $k \geq K$ then, 
  \begin{equation*}
    |x_N(k) - 0| < \epsilon/2.
  \end{equation*}
% We have found a sequence \epsilon/2 close to x_n, and called it x_N now we need to find out how far into x_N we need to go in order to start seeing zeros. 

  Let $\epsilon > 0$ and note that for $k \geq K$, by the triangle inequality it follows that, 
  \begin{align*}
    |x(k) - 0| &\leq |x(k) - x_N(k)| + |x_N(k) - 0|,\\
    &\leq \epsilon/2 + \epsilon/2,\\
    &= \epsilon. 
  \end{align*}



  
  %lim_{k \to \infty} x(k)= 0
  %wts
  %
  %|x|_{\infty} = \max_{k}\{|x(k)|\} < \epsilon
  %have
  %
  %lim_{k \to \infty} x_n(k) = 0
  %by convergence
  %\lim_{n \to \infty}|x_n - x|_\infty = \lim_{n \to \infty} \max_k\{|x_n(k) - x(k)|\} = 0
 %
  %||x_n| - |x|| \leq|x_n - x|
  %-|x_n - x|_1 \leq |x_n(k)| - |x(k)| \leq |x_n(k) - x(k)|_1
%
  %-|x_n - x|_1 \leq |x_n|_1 - |x|_1 \leq |x_n - x|_1
%

\end{proof}
\vspace*{.15in}


\problem \textbf{Carothers 7.22} Let $D$ be a dense subset of a metric space $M$, and suppose that every Cauchy sequence from $D$ converges to some point of $M$. Prove that $M$
is complete. 

%Just use a 1/n net around the sequence$ 
% Go far enough out so that your a sequence is within epsilon/2 of the x sequence then go even further to get that your a sequence is within epsilon/2 of its converging point. 
\begin{proof} Suppose $(x_n) \subseteq M$ and that $(x_n)$ is a Cauchy sequence. Construct the following sequence $(a_n) \subseteq D$ such that $a_n \in B_{1/n}(x_n)$. We will proceed to show that $(a_n)$ is a Cauchy sequence. 

  Let $\epsilon > 0$ and choose $N_1$ such that if $n \geq N_1$, then $1/n < \epsilon/3$, and also choose $N_2$ such that if $n, m \geq N_2$ then $d(x_n, x_m) < \epsilon/3$, note that for $N = \max\{N_1, N_2\}$ applying the triangle inequality shows that, 
  \begin{align*}
  d(a_n, a_m) &\leq d(a_n, x_n) + d(a_m, x_m) + d(x_n, x_m),\\
  &< \epsilon.
  \end{align*}

  Since $(a_n)$ is Cauchy, by our hypothesis it converges $a_n \to a$ in $M$. Now we will show that $(x_n)$ converges to $a$, a point in $M$ satisfying completeness. 
  Let $\epsilon > 0$ and choose $N = \max\{N_1, N_2\}$ such that if $n\geq N_1$, then $d(a_n, a) < \epsilon/2$ and if $n\geq N_2$, then $1/n < \epsilon/2$, then we apply the triangle inequality to find
  \begin{align*}
    d(x_n, a) &\leq d(x_n, a_n) + d(a_n, a),\\
    &< \epsilon/2 + \epsilon/2 = \epsilon. 
  \end{align*}

  
\end{proof}
\vspace*{.15in}


\problem \textbf{Carothers 7.32} Use Theorem 7.12 to prove that $\ell_1$ is complete. 
\begin{theorem}
  A normed vector space $X$ is complete if and only if every absolutly summable series in $X$ is summable. That is, $X$ is complete if and only if $\sum_{n = 1}^{\infty}x_n$ converges in $X$ whenever $\sum_{n = 1}^{\infty}||x_n|| < \infty$. 
\end{theorem}

As a preliminary, note that parentheticals like $S(k)$ or $x_n(k)$ denote the $k^{th}$ term in a sequence of real numbers and subscripts like $x_n$ denote the $n^{th}$ sequence in a sequence of sequences. 

\begin{proof} Let $(x_n) \subseteq \ell_1$ and suppose $\sum_{n = 1}^{\infty}||x_n||_1 < \infty$. Now to show $\ell_1$ is complete using Theorem 7.12 we must show that $\sum_{n = 1}^{\infty} x_n$ converges in $\ell_1$.

  We will first construct a candidate limit, let $S$ be the sequence of real numbers defined by
  \begin{equation*}
    S(k) = \sum_{n = 1}^{\infty} x_n(k). 
  \end{equation*}
  Note each term in $S$ is in fact a real number and bounded above by $S(k) < \sum_{n = 1}^{\infty}||x_n||_1 < \infty$. Now we will demonstrate that $S \in \ell_1$. Note that by definition, 
  \begin{equation*}
    \sum_{k  = 1}^\infty |S(k)| =  \sum_{k  = 1}^\infty |\sum_{n = 1}^{\infty} x_n(k)| \leq  \sum_{k  = 1}^\infty \sum_{n = 1}^{\infty} |x_n(k)|.
  \end{equation*}
  Note that by our hypothesis the following series converges absolutely,
  \begin{equation*}
    \sum_{n  = 1}^\infty \sum_{k = 1}^{\infty} |x_n(k)| = \sum_{n = 1}^{\infty}||x_n|| < \infty,
  \end{equation*}
  and therefore any rearrangement of the series will converge to the same limit. Therefore it follows that 
  \begin{equation*}
    \sum_{k  = 1}^\infty |S(k)| \leq \sum_{k  = 1}^\infty \sum_{n = 1}^{\infty} |x_n(k)| = \sum_{n  = 1}^\infty \sum_{k = 1}^{\infty} |x_n(k)| < \infty. 
  \end{equation*}
  Hence $S \in \ell_1$. 

  We will proceed to show that $\sum_{n = 1}^{\infty}x_n = S$. We define the sequence of partial sums by the following, 
  \begin{equation*}
    S_m = \sum_{n = 1}^m x_n.
  \end{equation*}
  Now we must show that $S_m \to S$ in $\ell_1$. Let $\epsilon > 0$, then choose $M$ such that $\sum_{M}^{\infty} ||x_n||_1 < \epsilon$, note that by our hypothsis this sum is the residual of a convergent series and can be made as small as desired, so then it follows that for all $m \geq M$,
  \begin{align*}
    ||S - S_m || &= \sum_{k}^\infty |S(k) - S_m(k)|\\
    &=\sum_{k}^\infty \lvert\sum_{n = 1}^{\infty} x_n(k) - \sum_{n = 1}^m x_n(k)\rvert \\
    &=\sum_{k}^\infty \lvert \sum_{n = m+1}^{\infty} x_n(k)\rvert\\
    &\leq \sum_{k}^\infty \sum_{n = m+1}^{\infty} |x_n(k)|\\
    &= \sum_{n = m+1}^{\infty} \sum_{k}^\infty |x_n(k)|\\
    &= \sum_{n = m+1}^{\infty} ||x_n||_1\\
    &< \epsilon.
  \end{align*} 
  Therefore $\sum_{n = 1}^{\infty}x_n = S$. Now since $\sum_{n = 1}^{\infty}x_n$ converges whenever $\sum_{n = 1}^{\infty}||x_n|| < \infty$ we apply Theorem 7.12 to conclude that $\ell_1$ is complete. 
  %First we will exhibit a candidate limit (the type of object that the limit is, is a sequence). 

  %Then we show that the sequence of sequences actually converges to that limit. 

  
\end{proof}
\vspace*{.15in}

%_________________________________________________________________________
%CHAPTER 8


% A and B are characterized by sequential compactness inhereted by the metric on M, 
% A, d is compact if and only if every sequence in A has a subsequence that converges to 
% a point in A,
% Choose a sequence in A \cup B. Show that it has a subsequence that converges in A\cup B
% (x_n) \subseteq A \cup B. Case 1 there exists a tail of the sequence is contained entirely in either A or B, in which case take the subsequence contained in one of the sets WLOG A. Since A is compact there is a further subsequence which converges to x in A

% Case 2 every tail of x_n is contained in A Cup B. Let x_n_b be the subsequence of elements in B and x_n_a elements in A. They both have further subsequences which converge, But they need to converge to the same limit. Suppose not, let $x_n_b' \to b$ and $x_n_a' \to a$ where b \not\eq to a. Actually it doesn't mattter this is stupid 
 \problem \textbf{Carothers 8.4} If $A$ and $B$ are compact sets in $M$, show that $A \cup B$ is compact. 
\begin{proof} Let $(x_n) \subseteq A \cup B$. In order to show that $A \cup B$ is compact we must demonstrate a subsequence of $(x_n)$ which converges to a value in $A \cup B$. It is either the case that infinitely many terms of $(x_n)$ reside in $A$, $B$, or both $A$ and $B$ so
  without loss of generality suppose infinitely many of the terms of $(x_n)$ exists in $A$. 
  Denote such terms in $A$ via subsequence $(x_{n_k})$. Note that since $A$ is compact there exists a further subsequence $(x_{n_{k_j}})$ which converges to a value in $A$ and therefore also $A \cup B$. Thus we have found a subsequence of $(x_n)$ which converges to a value in $A \cup B$, as desired. 





  %Without loss of generality suppose $(x_n)_{n = k} \subseteq A$ for some $k \in \NN$. Now since $(x_n)_{n = k} \subseteq A$ and $A$ is compact there exists a convergent subsequence $(x_{n_j})_{n = k} \subseteq (x_n)_{n = k} \subseteq (x_n)$. 
%
  %Now suppose every tail of $(x_n)$ is not contained in $A$ or $B$. Therefore without loss of generality there exists a subsequence $(x_{n_k})$ contained entirely within $A$. Again since $A$ is compact $(x_{n_k})$ contains a further convergent subsequence $(x_{n_{k_j}}) \subseteq (x_{n_k}) \subseteq (x_n)$.
\end{proof}
\vspace*{.15in}







\problem \textbf{Carothers 8.13} Given $c(n) \geq 0$ for all $n$, prove that the set $A = \{x\in \ell_2: |x(n)| \leq c(n), n \geq 1\}$ is compact in $\ell_2$ if and only if $\sum_{n = 1}^{\infty}c(n)^2 < \infty$. 
\begin{proof} $(\leftarrow)$ Let $c$ be a sequence such that $c(n) \geq 0$ for all $n$ and suppose that the set $A = \{x\in \ell_2: |x(n)| \leq c(n), n \geq 1\}$ is compact in $\ell_2$. We will proceed to show that $c \in \ell_2$ and therefore $\sum_{n = 1}^{\infty}c_n^2 < \infty$ by exhibiting a sequence in $A$, a compact and therefore closed set, such that whcih converges to $c$ in $\ell_2$. 


  Consider the sequence $(x_n)$ where the terms are defined by the following, 
  $$x_n = (c(1), c(2), c(3), \dots, c(n), 0, \dots).$$ For a fixed $n$ consider the sequence $x_n$, and note that by construction $x_n(k) \to 0$ and by continuouity of the $x^2$ 
  we know that $(x_n(k))^2 \to 0$ so then it follows that $\sum_{k = 1}^{\infty}(x_n(k))^2 < \infty$ and $x_n \in \ell_2$. 
  
  By construction we know that $|x_n(k)| \leq c(k)$ for all $k \geq 1$.
  Hence $(x_n) \subseteq A$, a compact set. Now since $A$ is compact there exists a convergent subsequence $(x_{n_j})$. Since this $(x_{n_j})$ converges in $\ell_2$ it converges coordinatewise and must therefore converge to $c$ as desired. 
\end{proof}


\begin{proof} Let $c$ be a sequence such that $c(n) \geq 0$ for all $n$, and suppose that 
  $\sum_{n = 1}^{\infty}c(n)^2 < \infty$. Consider $A = \{x\in \ell_2: |x(n)| \leq c(n), n \geq 1\}$, let $(x_n) \subseteq A$ such that $x_n \to x$. Since $(x_n) \subseteq \ell_2$, a metric space we also have coordinatewise convergence. By our definition of $A$ is folows that for each $k$ we have $|x_n(k)| < c(k)$, or equivalently  $-c(k)< x_n(k) < c(k)$. Since $x_n(k) \to x(k)$ it follows that $-c(k) \leq x(k) \leq c(k)$, and therefore $|x(k)| \leq c(k)$ hence $x \in A$. Thus $A$ is closed and since $\ell_2$ is a metric space $A$ is complete. 
  
  
  Now we wish to show that $A$ is totally bounded. Let $\epsilon > 0$ and since $\sum_{n = 1}^{\infty}c(n)^2 < \infty$ there exists an $N$ such that, 
  \begin{equation*}
    \sum_{n = N+1}^{\infty}c(n)^2  < \epsilon/2.
  \end{equation*}
  Note that for all $x \in A$, since $|x(n)| \leq c(n)$ for all $x \geq 1$ it follows that, 
  \begin{equation*}
    \sum_{n = N+1}^{\infty}x(n)^2 \leq \sum_{n = N+1}^{\infty}c(n)^2 < \epsilon/2.
  \end{equation*}
  Now consider $\hat{A} \subset \RR^n$ defined by, 
  \begin{equation*}
    \hat{A} = \{{x(n)}_{n = 1}^N: x(n) \in A\}. 
  \end{equation*}
  Now consider $C = \max{c(n)}_{1 \leq n \leq N}$ and note that $\hat{A}$ is bounded by the closed set $[-C, C]^n$ since again $|x(n)| \leq c(n)$ for all $x \geq 1$. Note that $\RR^n$ is totally bounded so consider an $\epsilon/2$-net and intersect it with $\hat{A}$, call it $\hat{W}$. Now consider $W \subseteq \ell_2$ defined by 
  \begin{equation*}
    W = \{x\in \ell_2: \{x(n)\}_{n = 1}^N \in \hat{W} \text{and } \{x(n)\}_{n = N+1}^\infty = 0 \}
  \end{equation*}
  Clearly $W \subseteq A$. Now we will show that $W$ is a suitable $\epsilon$-net. 
  Let $x \in A$. 
  
  I'm not certain how we choose the finite number of elements to do the argument, however it is clear to me that we can choose an $N$ ( to rule them all) such that, 
  \begin{equation*}
    \sum_{k = N}^\infty |c(k)| < \epsilon/2
  \end{equation*}
  and that it then follows that this $N$ works to bound every element $x_n$ in the net, 
  \begin{equation*}
    \sum_{k = N}^\infty |x_n(k)|  \leq  \sum_{k = N}^\infty |c(k)| < \epsilon/2.
  \end{equation*}

  Oh my goodness, ok now it's clear to me that the elements in the net for $A$ come from the finite diminsional vector space. 

  So we bound EVERY element in $A$ by some $N$ so the residual of their squared sum is size less than $\epsilon/2$ then we reduce to a finite dimensional vector space $\RR^N$, this new set in $\RR^N$ is totally bounded so we produce an $\epsilon/2$ net. We bring that net back into $\ell_2$. Note that it is in $A$ again, and then we show that every element in $A$ is within $\epsilon$ of our net. The first $N$ terms will be within $\epsilon/2$ of each other, then the remaining residual sum will be within $\epsilon/2$. 
  
\end{proof}



   %In the forward direction we construct a sequence c_1,0 0 0 
  % Then c_1, c_2, 0, 0, 0,  and so on. 


  % in the backwards direction $\sum_{n = 1}^{\infty}c_n^2 < \infty$ gives us an $N$ for which the tails of all our sequences in $(x_n) \subseteq A$ are less than epsilon/2. Look at all the finite sequences that have the tails removed. That is a bounded set in R^N and therefore totally bounded. choose an epsilon /2 net in R^N extend it to R^\infty then $A$ will also be totally bounded within some \epsilon/2 of this net and then \epsilon/2 within the tail of the sequences. ???


  % WTS $\sum_{n = 1}^{\infty}c_n^2  = c < \infty$ which is equivalent to saying that $c_n$ converges to 0 and the sequence of partial sums converges to some point c. 
  % OR and more likely just show that $c_n$ is in l2 dingus










  % A is all the sequences in l2 such that their absolute value or d(x_n, 0) \leq c_n
  % Since A is compact all these sequences have convergent subsequences that converge in A. 
  % working backwards d(c_n, 0) < \epsilon 
  % but we can also write d(c_n, 0) \leq d(c_n, x_n) + d(x_n, 0) by triangle inequality, 
  % now all we need to do is bound d(c_n, x_n) for some x_n \in A . okay how do we construct x_n in A so that this happens Heres what i want. for all \epsilon > 0 there exists N such that if 

\vspace*{.15in}






\problem \textbf{Carothers 8.17} If $M$ is compact show that $M$ is also seperable. 
\begin{proof} Suppose $M$ is a compact metric space. By definition of compactness, $M$ is also a totally bounded metric space. We have shown in the previous homework that totally bounded metric spaces are separable.
\end{proof}
\vspace*{.15in}









\problem \textbf{Carothers 8.29} Let $M$ be a compact metric space and suppose that $f: M \to M$ satisfies $d(f(x), f(y)) < d(x, y)$ whenever $x \neq y$. Show that $f$ has a fixed point. 

\begin{proof} Let $M$ be a compact metric space and suppose $f: M \to M$ satisfies $d(f(x), f(y)) < d(x, y)$ whenever $x \neq y$. Consider the function $g: M \to \RR_{\geq 0}$ defined by $g(x) = d(x, f(x))$. Note $g$ is continuous since it is a composition of continuous function $f$, $d$ and the identity map. Since $M$ is compact and $g$ continuous we have shown that $g$ is bounded and attains its minimum value. Let $x_0 \in M$ such that $g(x_0)$ is the minimum value. Suppose for the sake of contradiction that $g(x_0)\neq 0$ then $d(x_0, f(x_0)) > 0$ and therefore $x_0 \neq f(x_0)$. Let $y = f(x_0)$ and note that by our hypothesis,
  \begin{equation*}
    d(y, f(y)) = d(f(x_0), f(y)) < d(x_0, y) = d(x_0, f(x_0)).
  \end{equation*}
  However since $x_0$ is a minimizer its clear that $d(y, f(y)) = g(y) > g(x_0) = d(x_0, f(x_0))$. This is a contradiction, so we conclude that $g(x_0) = 0$ noting that $d(x_0, f(x_0)) = 0$ implies $f(x_0) = x_0$ and hence $f$ contains a fixed point. 
\end{proof}

%Let $f:[a, b] \to [a, b]$ be continuous. Show that $f$ has a fixed point. try to prove this without applying to the intermediate value theorem. [Hint: First note that $f$ is continuous; next consider $g(x) = |x - f(x)|$].
%
%\begin{proof} Let $f:[a, b] \to [a, b]$ be continuous. Now consider $g: [a, b] \to \RR$
%  defined by $g(x) = |x - f(x)|$. Let $\epsilon > 0$ and choose $0< \delta < \epsilon$ such that if $|x - y|< \delta$ then $|f(x) - f(y)| < \epsilon/2$
%  \begin{align*}
%    |g(x) - g(y)| &= ||x - f(x)| - |y - f(y)|| \\
%    &\leq |x - f(x) - y - f(y)| \\
%    &= |x - y - f(x) - f(y)| \\
%    &\leq |x - y| + |f(x) - f(y)| < \epsilon. 
%  \end{align*}
%
%  Since $g: [a, b] \to \RR$ is continuous it follows that $g([a, b]) = [c, d]$ for some $c, d \in \RR$.
%
%
%\end{proof}




\vspace*{.15in}







\problem \textbf{Carothers 8.38} If $M$ is compact, prove that every lower semicontinuous function on $M$ is bounded below and attains a minimum value. 
\begin{proof} Let $M$ be a compact space and let $f: M \to \RR$ be lower semicontinuous. Consider the set $f(M) \subseteq \RR$ which achieves an infimum, call it $m$. Since $f(M)$ is nonempty there exists an $f(x_n) \to m$, where $(f(x_n)) \subseteq f(M)$. Note that since $(f(x_n)) \subseteq f(M)$ there exists some corresponding $(x_n) \subseteq M$. Since $M$ is compact it there exists a convergent subsequence $(x_{n_j}) \to x$. Since $f$ is lower semicontinuous, a previous homework shows that $f(x) \leq \lim\inf_{n \to \infty}f(x_{n_j})$. Note that since $f(x_{n_j})$ is a subsequence of $f(x_n)$ it follows that, 
  \begin{equation*}
    f(x) \leq \lim\inf_{n \to \infty}f(x_{n_j}) = \lim\inf_{n \to \infty}f(x_{n}) = m
  \end{equation*}
  Since $m = \inf(M)$ it must follow that $f(x) = m$. 
  
  % Since f(M) is a non empty subset of $\RR$ it must be bounded below by some infimum.
  % as a further consequence of the completeness of RR we know that there exists an $f(x_n) \subseteq f(M)$ such that $f(x_n) \to inf(f(M))$. Note that sequence corresponds to some sequence in the preimage $f^{-1}(f(x_n) = (x_n) \subseteq M$ Note that since $M$ is compact there exists a convergent subsequence $x_{n_j} \to x$ in $M$. Since $f$ is lower semi continuous $f(x) \leq \lim \inf_{n \to \infty} f(x_{n_j}) = infimum$











  % F is lower semicontinuous if for each real $\alpha$  the set $\{x \in M: f(x) \leq \alpha\}$
  % is closed in $M$

  % F is loser semicontinusou if and only if whenever $x_n \to x$ in $M$ we know that $f(x) \leq \lim \inf_{n \to \infty} f(x_n)$

  %WTS that for all there exists an $x \in M$ such that $f(x) \leq f(a)$ for all $a \in M$. 
  
\end{proof}
\vspace*{.15in}







\problem \textbf{Carothers 8.40} Let $M$ be compact and let $f: M \to M$ satisfy $d(f(x), f(y)) = d(x, y)$ for all $x, y \in M$. Show that $f$ is onto. 
\begin{proof} Let $M$ be compact and let $f: M \to M$ satisfy $d(f(x), f(y)) = d(x, y)$ for all $x, y \in M$. First note that clearly $f$ is continuous, choose $\delta = \epsilon$ and the argument follows trivially. Since $f$ is continuous $f(M)$ is compact and since $f(M) \subseteq M$, a metric space then $f(M)$ is closed and bounded. 


  Suppose for the sake of contradiction that $f$ is not onto. Then there exists an $x_0 \in M$ such that $x_0 \in f(M)^c$, an open set. Therefore there exists $B_\epsilon(x_0)$ such that $ B_\epsilon(x_0) \subseteq f(M)^c$. We define a new sequence $(x_n) \subseteq M$ by $x_{n} = f^n(x_0)$ where $f^n(x_0)$ denotes $n$ compositions of $f$ applied to $x_0$. Let $n < m$ and note that since $f$ is an isometry we find that, 
  \begin{equation*}
    d(x_n, x_m) = d(f^n(x_0), f^m(x_0)) = d(x_0,f^{m - n}(x_0)) > \epsilon, 
  \end{equation*}
  since $f^{m - n}(x_0) \in f(M)$ and $B_\epsilon(x_0) \subseteq f(M)^c$. Therefore we have shown that $(x_n)$ is not Cauchy, however since $(x_n) \subseteq M$ which is a compact space, there exists a convergent subsequence $(x_{n_j})$ which must be Cauchy, a contradiction. 
  
  
\end{proof}
\vspace*{.15in}






\end{problems}
\end{document}
