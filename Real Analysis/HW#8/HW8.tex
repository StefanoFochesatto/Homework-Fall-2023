\documentclass[12pt]{article}

\usepackage{amssymb,amsmath,amsthm}
\usepackage[top=1in, bottom=1in, left=1.25in, right=1.25in]{geometry}
\usepackage{fancyhdr}
\usepackage{mathtools}
\usepackage{enumerate}
\usepackage{color}
\usepackage[english]{babel}
\newtheorem*{theorem}{Theroem 7.12}

%% Import a package useful for displaying graphics.
\usepackage{graphicx}

% Comment the following line to use TeX's default font of Computer Modern.
\usepackage{times,txfonts}

% Adjust line spacing an indentation
\parindent 0pt
\parskip 6pt plus 1pt

\newtheoremstyle{ex215}% name of the style to be used
  {18pt}% measure of space to leave above the theorem. E.g.: 3pt
  {12pt}% measure of space to leave below the theorem. E.g.: 3pt
  {}% name of font to use in the body of the theorem
  {}% measure of space to indent
  {\bfseries}% name of head font
  {:}% punctuation between head and body
  {2ex}% space after theorem head; " " = normal interword space
  {}% Manually specify head
\theoremstyle{ex215} 

\makeatletter
\newtheorem*{lemmacore}{Lemma \@currentlabel}
\newenvironment{lemma}[1]
{\def\@currentlabel{#1}\lemmacore}
{\endlemmacore}

\newtheorem*{corollarycore}{Corollary \@currentlabel}
\newenvironment{corollary}[1]
{\def\@currentlabel{#1}\corollarycore}
{\endcorollarycore}


%%%%%%%%%%%%%%%%%%%%%%%%%%%%%%%%%%%%%%%%%%%%%%%%%%%%%%%%%%%%%%%%%%%%%%%%
%
% Stuff for getting the name/document date/title across the header
\RequirePackage{fancyhdr}
\pagestyle{fancy}
\fancyfoot[C]{\ifnum \value{page} > 1\relax\thepage\fi}
\fancyhead[L]{\ifx\@doclabel\@empty\else\@doclabel\fi}
\fancyhead[C]{\ifx\@docauthor\@empty\else\@docdate\fi}
\fancyhead[R]{\ifx\@docauthor\@empty\@docdate\else\@docauthor\fi}
\headheight 15pt


\def\doclabel#1{\gdef\@doclabel{#1}}
\doclabel{Use {\tt\textbackslash doclabel\{MY LABEL\}}.}
\def\docdate#1{\gdef\@docdate{#1}}
\docdate{Use {\tt\textbackslash docdate\{MY DATE\}}.}
\def\docauthor#1{\gdef\@docauthor{#1}}
%\docauthor{Use {\tt\textbackslash docauthor\{MY NAME\}}.}
\let\@docauthor\@empty

\newcounter{probcount}
\newcounter{subprobcount}
\newlength\probsep
\newlength\pshrinking
\newif\iffirstprob

% The following commands allow for a handy override of labels in an enumeration,
% without having the user keep track of what level the enumeration is happening at.
% To use them, just put \listlabel{....} as the first command after a \begin{enumerate}.
% The '....' is used to define the label.  You can access the name of the current label
% counter with \thelistlabel.  Hence \listlabel{\textbf{\Alph{\thelistlabel}:}} would
% generate lists with bold labels A:, B:, C:, etc.
\newcommand{\listlabel}[1]{\def\thelistlabel{\@enumctr}%
            \expandafter\def\csname label\@enumctr\endcsname{#1}}


\newenvironment{subproblems}%
  {\begin{enumerate}\listlabel{\alph{\thelistlabel})}%
  %% Make the enumerate environment think we are making a list in a list:
  \global \@newlistfalse}%
  {\end{enumerate}}%

\newenvironment{problems}%
  {\ifhmode\unskip\par\fi\setcounter{probcount}{0}\probsep\parskip
  \sbox\@tempboxa{\textbf{9.}}\pshrinking\wd\@tempboxa\advance\pshrinking\labelsep
  \advance\linewidth -\pshrinking
  \advance\@totalleftmargin\pshrinking
  \advance\leftskip\pshrinking}%
  {\ifhmode\unskip \par\fi\advance\leftskip-\pshrinking}%

\newcommand{\problem}{%
  \setcounter{subprobcount}{0}%
  \stepcounter{probcount}%
  \def\@currentlabel{\arabic{probcount}}%
  \ifhmode
    \unskip \par
  \fi
%  \addpenalty{-4000}%
  \iffirstprob\else\addvspace\probsep\fi
  \firstprobfalse
  \hskip -\labelwidth\hskip -\labelsep 
  \hbox to\labelwidth{\hss\textbf{\arabic{probcount}.}}\hskip\labelsep
}%

%% The localhead command puts a label on top of a paragraph -- handy for labels
%% like "Solution:"
\newskip\restoreparskip
\newcommand{\localhead}[1]{\par\smallskip\textbf{#1}\nobreak\\}%
%% The following arcane code was added to make the localhead enviroment get along
%% with the ams theorem enviroment (which is based on the latex list enviroment).
%% previously, the command was {\par\smallskip\textbf{#1}\\}, but the list
%% enviroment would also add a par at the end of this paragraph, which would result
%% in an extra blank line.  Our solution now is to put in our own par, but with
%% none of the parskip glue that adds extra space between paragraphs.
%%  \restoreparskip\parskip\parskip 0pt\par\nobreak\noindent\parskip\restoreparskip}%
\newcommand\solution{\localhead{Solution:}}
\newcommand\subsol{\stepcounter{subprobcount}\localhead{Solution, part \alph{subprobcount}:}}
\newcommand\subsoleg[1]{\stepcounter{subprobcount}\par\textbf{Solution, part \alph{subprobcount}:} #1\\}
\def\solver#1{(Solution by #1)\\\vskip-12pt}


%% The default 'article' class has captions that label figures Figure 1, etc.  This is too 
%% formal for homework sets, so we change \caption so that it just puts the caption text.
%% Here I have just modified the \@makecaption command from the article class.
%% Maybe I should change this in the future to let authors decide whether or not to label a figure.

\long\def\@makecaption#1#2{%
  \vskip\abovecaptionskip
  \sbox\@tempboxa{#2}%
  \ifdim \wd\@tempboxa >\hsize
    #2\par
  \else
    \global \@minipagefalse
    \hb@xt@\hsize{\hfil\box\@tempboxa\hfil}%
  \fi
  \vskip\belowcaptionskip}

%% The default implementation of the proof environment starts a new paragraph at the start of the proof,
%% with the full parskip spacing. This only looks good following a theorem statement 
%% if \parskip is set to zero.  We've got a medium sized parskip, so we overide this 
%% questionable decision.
\renewenvironment{proof}[1][\proofname]{\par
  \pushQED{\qed}%
  \normalfont \topsep6\p@\@plus6\p@\relax
  \trivlist
  \@topsep \topsep
  \item[\hskip\labelsep
        \itshape
    #1\@addpunct{.}]\ignorespaces
}{%
  \popQED\endtrivlist\@endpefalse
}
\makeatother

% Shortcuts for blackboard bold number sets (reals, integers, etc.)
\newcommand{\Reals}{\ensuremath{\mathbb R}}
\newcommand{\Nats}{\ensuremath{\mathbb N}}
\newcommand{\Ints}{\ensuremath{\mathbb Z}}
\newcommand{\Rats}{\ensuremath{\mathbb Q}}
\newcommand{\Cplx}{\ensuremath{\mathbb C}}
\newcommand{\diam}[1]{\text{diam}(#1)}
\newcommand\converges[1]{\mathrel{\mathop{\longrightarrow}\limits_{#1}}}
\newcommand{\norm}[2]{\left \lVert #1 \right \rVert_{#2}}
\newcommand{\abs}[1]{\left| #1 \right|}
\newcommand{\FF}{\mathcal{F}}



\let\latexchi\chi
\makeatletter
\renewcommand\chi{\@ifnextchar_\sub@chi\latexchi}
\newcommand{\sub@chi}[2]{% #1 is _, #2 is the subscript
  \@ifnextchar^{\subsup@chi{#2}}{\latexchi^{}_{#2}}%
}
\newcommand{\subsup@chi}[3]{% #1 is the subscript, #2 is ^, #3 is the superscript
  \latexchi_{#1}^{#3}%
}
\makeatother

%% Some equivalents that some people may prefer.
\let\RR\Reals
\let\NN\Nats
\let\II\Ints
\let\CC\Cplx
\let\ZZ\Ints

\doclabel{Math F641: Homework \#8}
\docauthor{Stefano Fochesatto}
\docdate{\today}
\begin{document}
This was also a very poor showing, very sorry. 
\begin{problems}

  \problem \textbf{Carothers 11.65} Let $K(x, t)$ be a continuous function on the square $[a, b] \times [a, b]$. 
  \begin{enumerate}
    \item[\textbf{(a)}] Given $f \in C[a, b]$, show that $g(x) = \int_{a }^{b }f(t) K(x, t)dt$ defines a continuous function $g \in C[a, b]$.

    \begin{proof} Let $\epsilon > 0$. 
      Note that since $K$ and $f$ are continuous functions it follows that $f(x)K(x, t)$ is also continuous on $[a, b] \times [a, b]$. Since $f(x)K(x, t)$ is a continuous function on a compact domain, it is also uniformly continuous, therefore there exists a $\delta$, such that for all $c,d \in [a, b] \times [a, b]$  where $0 < \max{|x_c - x_d|, |t_c- t_d|} < \delta$ we get, 
      \begin{equation*}
        |f(x_c)f(x_c,t_c) - f(x_d)f(x_d,t_d)| < \epsilon.
      \end{equation*}

      Then for a fixed $t \in [a, b]$ it follows that if $0 < \abs{x - y} < \delta$, clearly we get that $\max{|x - y|, |t- t| = 0} = \abs{x - y} < \delta$.
      \begin{align*}
        \abs{\int_{a }^{b }f(t) K(x, t)dt - \int_{a }^{b }f(t) K(y, t)dt} &= \abs{\int_{a }^{b }f(t)K(x, t) - f(t)K(y, t)dt}\\
        &\leq\int_{a }^{b }\abs{f(t)K(x, t) - f(t)K(y, t)}dt\\
        &<\int_{a }^{b }\epsilon dt =  (b - a) \epsilon.
      \end{align*}
 



      \end{proof}
      \vspace*{.15in}

      \item[\textbf{(b)}] Define $T: C[a, b] \to C[a, b]$ by $(Tf)(x)= \int_{a }^{b } f(t) K(x, t)dt$. Show that $T$ maps bounded sets into equicontinuous sets. In particular, $T$ is continuous. 
      \begin{proof} Let $\FF \subset C[a, b]$ be uniformly bounded. We wish to prove that $T(\FF)$ is equicontinuous, and we will proceed almost exactly as the previous problem. Let $\epsilon > 0$. Since $\FF$ is uniformly bounded there exists an $f_m$ such that $|f(t)| \leq f_m$ for all $f \in \FF$ and $t \in [a,b]$. Now again, $f_mK(x, t)$ is continuous on a compact domain, it is also uniformly continuous. Therefore there exists a $\delta$, such that for all $c,d \in [a, b] \times [a, b]$  where $0 < \max\{|x_c - x_d|, |t_c- t_d|\} < \delta$ we get, 
      \begin{equation*}
        |f_mK(x_c,t_c) - f_mK(x_d,t_d)| = |f_m\left(K(x_c,t_c) - K(x_d,t_d)\right)| < \epsilon.
      \end{equation*}
      Fix $t \in [a, b]$ and we find that when $0 < |x - y|  < \delta$ clearly we get that $\max\{|x - y|, |t- t| = 0\} = \abs{x - y} < \delta$ so it follows that,
        \begin{align*}
          \abs{\int_{a }^{b }f(t) K(x, t)dt - \int_{a }^{b }f(t) K(y, t)dt} &=  \abs{\int_{a }^{b }f(t)K(x, t) - f(t)K(y, t)dt}\\
          &=\abs{\int_{a }^{b }f(t)(K(x, t) -K(y, t))dt}\\
          &\leq\int_{a }^{b }\abs{f(t)}\abs{(K(x, t) -K(y, t))}dt\\
          &\leq\int_{a }^{b }\abs{f_m}\abs{K(x, t) - K(y, t)}dt\\
          &< (b - a) \epsilon.
        \end{align*}
        Since we've found a single $\delta$ which satisfies the $\epsilon-\delta$ definition of uniform continuity for all $f \in \FF$ simultaneously we can conclude that $\FF$ is equicontinuous.

        Now note that $T$ is a linear operator, since for $f, g \in C[a, b]$ and $\alpha , \beta \in \RR$ we find that, 
        \begin{align*}
          T(\alpha f + \beta g) &= \int_{a }^{b } \left(\alpha f(t) + \beta g(t) \right) K(x, t)dt\\ 
          &= \int_{a }^{b }\alpha f(t)K(x, t) + \beta g(t)K(x, t) dt \\
          &= \alpha \int_{a }^{b }f(t)K(x, t)dt+ \beta \int_{a }^{b } g(t)K(x, t) dt = \alpha T(f) + \beta T(g)
        \end{align*}
        Now let $f \in C[a, b]$, and since $K(x, t)$ is a continuous function on the compact domain $[a, b] \times [a, b]$ so $K$ achieves a maximum at some $K_M$. Finally we show boundedness with, 
        \begin{align*}
          \norm{T(f)}{\infty} &= \norm{\int_{a }^{b } f(t) K(x, t)dt}{\infty} \leq \norm{\int_{a }^{b } \norm{f}{\infty} K_M dt}{\infty}\\
           &= \norm{\norm{f}{\infty} \int_{a }^{b } K_M dt}{\infty}\\
           &= \norm{f}{\infty} \norm{\int_{a }^{b } K_M dt}{\infty}\\
           &= \norm{f}{\infty} (b - a)K_m
        \end{align*}
      \end{proof}


      \vspace*{.15in}

      \item[\textbf{(c)}] Show that if $\int_{a }^{b }\abs{K(x, t )}dt \leq 1$ for all $x \in [a, b]$ then the Arzela-Ascoli Theorem implies that given any $f \in C[a, b]$, the sequence $(T^{(n)}f)_n$ has a subsequence that converges in $C[a, b]$. 
      \begin{proof} Let $\FF$ be the set $(T^{(n)}f)_n$. We will show that $T(\FF)$ is equicontinuous by proving that $(T^{(n)}f)_n$ is uniformly bounded, and applying the previous result. Since $f \in C[a, b]$ we know that there exists an $f_m$ such that $\abs{f(x)} < f_m$ for all $x \in [a, b]$. We will proceed by induction to show that $\FF$ is uniformly bounded by $f_m$. Note that $T^{(1)}f$ can be written as the following, 
        \begin{equation*}
          |T^{(1)}f| = \abs{\int_{a }^{b }f(x)K(x, t)dt} \leq \int_{a }^{b }\abs{f(x)}\abs{K(x, t)}dt \leq f_m  \int_{a }^{b }\abs{K(x, t)}dt \leq f_m. 
        \end{equation*}
        Now suppose $|T^{(n)}f| \leq f_m$ and consider $T^{(n + 1)}$ can be written as the following, 
        \begin{equation*}
          |T^{(n + 1)}| = \abs{\int_{a }^{b } (T^{(n)}f(x)) K(x, t) dt} \leq \int_{a }^{b } \abs{T^{(n)}f(x)} \abs{K(x, t)} dt \leq f_m \int_{a }^{b }\abs{K(x, t)}dt \leq f_m. 
        \end{equation*}
        Thus we have show that $\FF$ is uniformly bounded and by the previous result we know that $T(\FF)$ is an equicontinuous set of functions. It also follows that since $T(\FF) \subseteq \FF$ we know that $T(\FF)$ is uniformly bounded. Now consider $\overline{T(\FF)}$ a closed, bounded, equicontinuous set (the limit of a uniformly convergent sequence of equicontinuous functions must be uniformly continuous by the same $\delta$, and its also probably bounded. Definitely need to spend some more time on this result.), we conclude be Arzela-Ascoli that $\overline{T(\FF)}$ is compact. Since $T(\FF)$ is a sequence contained in a compact set, $\overline{T(\FF)}$ it must have a convergent subsequence, and therefore since $T(\FF) \subseteq \FF$, we can conclude that $\FF$ has a convergent subsequence. 
      \end{proof}
      \vspace*{.15in}
  \end{enumerate}
  \vspace*{.5in}






  \problem Suppose $f \in \mathcal{R}[a, b]$ and $\alpha \in \RR$. Show that $\alpha f \in \mathcal{R}[a, b]$ and, 
  \begin{equation*}
    \int_{a }^{b }\alpha f = \alpha \int_{a }^{b }f.
  \end{equation*}
  \begin{proof} First we will show that $\alpha f \in \mathcal{R }[a, b]$. Since $f \in \mathcal{R }[a, b]$ there exists step functions $H_n$ with $H_n \geq f$ such that, 
    \begin{equation*}
      \int_{a }^{b }H_n \leq \int_{a }^{b }f + \frac{1}{n}. 
    \end{equation*}
    Similarly we there exists step functions $h_n$ such that $h_n \leq f$ with the analogous property that, 
    \begin{equation*}
      \int_{a }^{b }h_n \geq \int_{a }^{b }f - \frac{1}{n}. 
    \end{equation*}
    Clearly $\int_{a }^{b }H_n \to \int_{a }^{b }f$ and $\int_{a }^{b }h_n \to \int_{a }^{b }f$. 
    So for $\alpha > 0$ it follows that for all $x \in [a, b]$, 
    \begin{equation*}
      \alpha h_n(x) \leq \alpha f(x) \leq \alpha H_n(x).
    \end{equation*}
    Integrating over $[a, b]$ we find that
    \begin{equation*}
      \int_{a }^{b } \alpha h_n(x) \leq  \int_{a }^{b } \alpha f(x) \leq  \int_{a }^{b } \alpha H_n(x).
    \end{equation*}
    Since $H_n$ and $h_n$ are step functions by linearity it follows that, 
    \begin{equation*}
      \alpha \int_{a }^{b }f - \frac{1}{n}\leq \alpha \int_{a }^{b } h_n(x) \leq  \int_{a }^{b } \alpha f(x) \leq  \alpha \int_{a }^{b } H_n(x) \leq \alpha \int_{a }^{b }f + \frac{1}{n}.
    \end{equation*}
    Taking the limit we find that, 
    \begin{equation*}
      \alpha \int_{a }^{b }f \leq  \int_{a }^{b } \alpha f(x) \leq \alpha \int_{a }^{b }f.
    \end{equation*}
    So we conclude that, 
    $\alpha f \in \mathcal{R}[a, b]$ and, 
  \begin{equation*}
    \int_{a }^{b }\alpha f = \alpha \int_{a }^{b }f.
  \end{equation*}
    
  \end{proof}
  \vspace*{.5in}



  \problem Show that the uniform limit of Riemann integrable functions is Riemann integrable. Conclude that $\mathcal{R}[a, b]$ is a closed subspace of $B[a, b]$. 
  \begin{proof}  let $f_n \to f$ converge uniformly such that $f_n \in \mathcal{R}[a, b]$. Let $\epsilon > 0$. Since  $f_n \to f$ choose $N$ such that, $\abs{f(x) - f_N(x)} < \epsilon$ for all $x \in [a, b]$. Now since $f_N \in \mathcal{R}[a, b]$ there exists $H, h \in \text{Step}[a, b]$ with $h \leq f_N \leq H$ such that,
    \begin{equation*}
      \int_{a }^{b }H(x) - h(x) dx < \epsilon. 
    \end{equation*} 

    Now note that since $\abs{f(x) - f_N(x)} < \epsilon$ it follows that, 
    \begin{align*}
      -\epsilon < &f(x) - f_N(x) < \epsilon,\\
      f_N(x) - \epsilon < &f(x) < f_N(x) + \epsilon,\\
      h(x) - \epsilon < &f(x) < H(x) + \epsilon.
    \end{align*}
    We see that a step function plus a constant is simply another step function so $h(x) - \epsilon , H(x) + \epsilon \in \text{Step}[a, b]$ and we also find that by linearity of the integral of step functions, 
    \begin{align*}
      \int_{a }^{b }( H(x) + \epsilon) -( h(x) - \epsilon) dx &= \int_{a }^{b }H(x) - h(x)  + 2\epsilon dx,\\
      &=\int_{a }^{b }H(x) - h(x)dx  + \int_{a }^{b }2\epsilon dx,\\
      &< \epsilon  + (b - a)2\epsilon = (1 + 2b - 2a)\epsilon.
    \end{align*}
    Therefore $f \in \mathcal{R}[a, b]$. Recall that by definition $\mathcal{R}[a, b] \subseteq B[a, b]$ and therefore by our result it follows that $\mathcal{R}[a, b]$ is a closed subspace of $B[a, b]$. 
    
    
  \end{proof}
  \vspace*{.5in}


  \problem Determine if $\chi_{\Delta} \in \mathcal{R}[0, 1]$, where $\Delta$ is the Cantor set. 
  \begin{proof} 
  Recall the that $\Delta$ can be defined by the following recurrence relation on sets. 
   \begin{align*}
    I_0 &= [0, 1]\\
    I_{k + 1} &= \frac{1}{3} I_k \cup \left(\frac{2}{3} + \frac{1}{3}I_k\right)\\
    \Delta &\coloneqq \bigcap I_k
  \end{align*}
  Now let $H_n$ be the characteristic function on the set $I_k$. We know that $H_n \in \text{Step}[a, b]$ since each $I_k$ is a union of finitely many intervals. Also we have that $H_n \geq \chi(\Delta)$ since $\Delta \subset I_k$. Since $H_n$ is the characteristic function on the set $I_k$ its clear that, 
  \begin{equation*}
    \int_{0 }^{1 }H_n = \left(\dfrac{2}{3}\right)^n.
    \end{equation*}
    Note that this integral converges to zero. Therefore we know that, 
    \begin{equation*}
      \overline{\int_{0}^{1}} \chi(\delta) \leq \int_{0 }^{1 }H_n \to 0.
    \end{equation*}
    Now clearly we can consider an $h \in \text{Step}[0, 1]$ where $h = 0$ and therefore $h \leq \chi{\Delta}$. So it follows, that 
    \begin{equation*}
      \underline{\int_{0}^{1}} \chi(\delta) \geq 0.
    \end{equation*}
    Thus $\chi_{\Delta} \in \mathcal{R}[0, 1]$ and $\int_{0}^{1} \chi(\delta) = 0$. 

     
  \end{proof}






\end{problems}
\end{document}
