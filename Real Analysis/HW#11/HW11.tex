\documentclass[12pt]{article}

\usepackage{amssymb,amsmath,amsthm}
\usepackage[top=1in, bottom=1in, left=1.25in, right=1.25in]{geometry}
\usepackage{fancyhdr}
\usepackage{mathtools}
\usepackage{enumerate}
\usepackage{color}
\usepackage[english]{babel}
\newtheorem*{theorem}{Theroem 7.12}

%% Import a package useful for displaying graphics.
\usepackage{graphicx}

% Comment the following line to use TeX's default font of Computer Modern.
\usepackage{times,txfonts}

% Adjust line spacing an indentation
\parindent 0pt
\parskip 6pt plus 1pt

\newtheoremstyle{ex215}% name of the style to be used
  {18pt}% measure of space to leave above the theorem. E.g.: 3pt
  {12pt}% measure of space to leave below the theorem. E.g.: 3pt
  {}% name of font to use in the body of the theorem
  {}% measure of space to indent
  {\bfseries}% name of head font
  {:}% punctuation between head and body
  {2ex}% space after theorem head; " " = normal interword space
  {}% Manually specify head
\theoremstyle{ex215} 

\makeatletter
\newtheorem*{lemmacore}{Lemma \@currentlabel}
\newenvironment{lemma}[1]
{\def\@currentlabel{#1}\lemmacore}
{\endlemmacore}

\newtheorem*{corollarycore}{Corollary \@currentlabel}
\newenvironment{corollary}[1]
{\def\@currentlabel{#1}\corollarycore}
{\endcorollarycore}


%%%%%%%%%%%%%%%%%%%%%%%%%%%%%%%%%%%%%%%%%%%%%%%%%%%%%%%%%%%%%%%%%%%%%%%%
%
% Stuff for getting the name/document date/title across the header
\RequirePackage{fancyhdr}
\pagestyle{fancy}
\fancyfoot[C]{\ifnum \value{page} > 1\relax\thepage\fi}
\fancyhead[L]{\ifx\@doclabel\@empty\else\@doclabel\fi}
\fancyhead[C]{\ifx\@docauthor\@empty\else\@docdate\fi}
\fancyhead[R]{\ifx\@docauthor\@empty\@docdate\else\@docauthor\fi}
\headheight 15pt


\def\doclabel#1{\gdef\@doclabel{#1}}
\doclabel{Use {\tt\textbackslash doclabel\{MY LABEL\}}.}
\def\docdate#1{\gdef\@docdate{#1}}
\docdate{Use {\tt\textbackslash docdate\{MY DATE\}}.}
\def\docauthor#1{\gdef\@docauthor{#1}}
%\docauthor{Use {\tt\textbackslash docauthor\{MY NAME\}}.}
\let\@docauthor\@empty

\newcounter{probcount}
\newcounter{subprobcount}
\newlength\probsep
\newlength\pshrinking
\newif\iffirstprob

% The following commands allow for a handy override of labels in an enumeration,
% without having the user keep track of what level the enumeration is happening at.
% To use them, just put \listlabel{....} as the first command after a \begin{enumerate}.
% The '....' is used to define the label.  You can access the name of the current label
% counter with \thelistlabel.  Hence \listlabel{\textbf{\Alph{\thelistlabel}:}} would
% generate lists with bold labels A:, B:, C:, etc.
\newcommand{\listlabel}[1]{\def\thelistlabel{\@enumctr}%
            \expandafter\def\csname label\@enumctr\endcsname{#1}}


\newenvironment{subproblems}%
  {\begin{enumerate}\listlabel{\alph{\thelistlabel})}%
  %% Make the enumerate environment think we are making a list in a list:
  \global \@newlistfalse}%
  {\end{enumerate}}%

\newenvironment{problems}%
  {\ifhmode\unskip\par\fi\setcounter{probcount}{0}\probsep\parskip
  \sbox\@tempboxa{\textbf{9.}}\pshrinking\wd\@tempboxa\advance\pshrinking\labelsep
  \advance\linewidth -\pshrinking
  \advance\@totalleftmargin\pshrinking
  \advance\leftskip\pshrinking}%
  {\ifhmode\unskip \par\fi\advance\leftskip-\pshrinking}%

\newcommand{\problem}{%
  \setcounter{subprobcount}{0}%
  \stepcounter{probcount}%
  \def\@currentlabel{\arabic{probcount}}%
  \ifhmode
    \unskip \par
  \fi
%  \addpenalty{-4000}%
  \iffirstprob\else\addvspace\probsep\fi
  \firstprobfalse
  \hskip -\labelwidth\hskip -\labelsep 
  \hbox to\labelwidth{\hss\textbf{\arabic{probcount}.}}\hskip\labelsep
}%

%% The localhead command puts a label on top of a paragraph -- handy for labels
%% like "Solution:"
\newskip\restoreparskip
\newcommand{\localhead}[1]{\par\smallskip\textbf{#1}\nobreak\\}%
%% The following arcane code was added to make the localhead enviroment get along
%% with the ams theorem enviroment (which is based on the latex list enviroment).
%% previously, the command was {\par\smallskip\textbf{#1}\\}, but the list
%% enviroment would also add a par at the end of this paragraph, which would result
%% in an extra blank line.  Our solution now is to put in our own par, but with
%% none of the parskip glue that adds extra space between paragraphs.
%%  \restoreparskip\parskip\parskip 0pt\par\nobreak\noindent\parskip\restoreparskip}%
\newcommand\solution{\localhead{Solution:}}
\newcommand\subsol{\stepcounter{subprobcount}\localhead{Solution, part \alph{subprobcount}:}}
\newcommand\subsoleg[1]{\stepcounter{subprobcount}\par\textbf{Solution, part \alph{subprobcount}:} #1\\}
\def\solver#1{(Solution by #1)\\\vskip-12pt}


%% The default 'article' class has captions that label figures Figure 1, etc.  This is too 
%% formal for homework sets, so we change \caption so that it just puts the caption text.
%% Here I have just modified the \@makecaption command from the article class.
%% Maybe I should change this in the future to let authors decide whether or not to label a figure.

\long\def\@makecaption#1#2{%
  \vskip\abovecaptionskip
  \sbox\@tempboxa{#2}%
  \ifdim \wd\@tempboxa >\hsize
    #2\par
  \else
    \global \@minipagefalse
    \hb@xt@\hsize{\hfil\box\@tempboxa\hfil}%
  \fi
  \vskip\belowcaptionskip}

%% The default implementation of the proof environment starts a new paragraph at the start of the proof,
%% with the full parskip spacing. This only looks good following a theorem statement 
%% if \parskip is set to zero.  We've got a medium sized parskip, so we overide this 
%% questionable decision.
\renewenvironment{proof}[1][\proofname]{\par
  \pushQED{\qed}%
  \normalfont \topsep6\p@\@plus6\p@\relax
  \trivlist
  \@topsep \topsep
  \item[\hskip\labelsep
        \itshape
    #1\@addpunct{.}]\ignorespaces
}{%
  \popQED\endtrivlist\@endpefalse
}
\makeatother

% Shortcuts for blackboard bold number sets (reals, integers, etc.)
\newcommand{\Reals}{\ensuremath{\mathbb R}}
\newcommand{\Nats}{\ensuremath{\mathbb N}}
\newcommand{\Ints}{\ensuremath{\mathbb Z}}
\newcommand{\Rats}{\ensuremath{\mathbb Q}}
\newcommand{\Cplx}{\ensuremath{\mathbb C}}
\newcommand{\diam}[1]{\text{diam}(#1)}
\newcommand\converges[1]{\mathrel{\mathop{\longrightarrow}\limits_{#1}}}
\newcommand{\norm}[2]{\left \lVert #1 \right \rVert_{#2}}
\newcommand{\abs}[1]{\left| #1 \right|}
\newcommand{\FF}{\mathcal{F}}
\newcommand{\A}{\mathcal{A}}
\newcommand{\BB}{\mathcal{B}}


\newcommand{\QQ}{\ensuremath{\mathbb Q}}



\let\latexchi\chi
\makeatletter
\renewcommand\chi{\@ifnextchar_\sub@chi\latexchi}
\newcommand{\sub@chi}[2]{% #1 is _, #2 is the subscript
  \@ifnextchar^{\subsup@chi{#2}}{\latexchi^{}_{#2}}%
}
\newcommand{\subsup@chi}[3]{% #1 is the subscript, #2 is ^, #3 is the superscript
\latexchi_{#1}^{#3}%
}
\makeatother

%% Some equivalents that some people may prefer.
\let\RR\Reals
\let\NN\Nats
\let\II\Ints
\let\CC\Cplx
\let\ZZ\Ints

\doclabel{Math F641: Homework \#11}
\docauthor{Stefano Fochesatto}
\docdate{\today}
\begin{document}



\begin{problems}

  \problem \textbf{Carothers 17.3} Let $f : D \to \RR $, where $D$ is measurable. Show that $f $ is measurable if and only if the function $g : \RR \to \RR $ is measurable, where $g(x ) = f(x)$ for $x \in D $ and $g(x ) = 0$ for $x \not\in D$.    
  \begin{proof} Let $f: D \to \RR$, where $D$ is measurable and suppose that $f$ is measurable. Define the function $g : \RR \to \RR$ such that $g(x) = f(x)$ when $x \in D$ and $g(x) = 0$ otherwise. Let $\alpha \in \RR$ and consider the set $\{g \geq \alpha\}$, and note that if $\alpha > 0$ then $\{g \geq \alpha\} = \{f \geq \alpha\}$ which is measurable, since $f$ is measurable. Now if $\alpha \leq 0$ we find that $\{g \geq \alpha\} = \{f \geq \alpha\} \cup D^c$ a union of measurable sets, which is also measurable. Hence $g$ is a measurable function.
  \end{proof}




  \begin{proof}  Let $f : D \to \RR $, where $D$ is measurable and suppose that the function $g : \RR \to \RR $ is measurable, where $g(x) = f(x)$ for $x \in D $ and $g(x) = 0$ for $x \not\in D$. Note that $\{f \geq \alpha \} = \{g \geq \alpha \} \cap D$ which is always measurable. 

  \end{proof}
  \vspace*{.15in}






  \problem \textbf{Carothers 17.4} Prove that $\chi_E$ is measurable if and only if $E$ is measurable. 
  \begin{proof} Suppose $\chi_E$ is measurable, then clearly it follows that $\{\chi_E \geq \frac{1}{2}\} = E$ and must also be measurable. Now suppose that $E$ is measurable, and consider the function $\chi_E$. Note that
    \begin{equation*}
      \{\chi_E \geq \alpha \}= \begin{cases}
        \emptyset & \alpha > 1\\
        E &  0 < \alpha \leq 1\\
        \RR &   \alpha \leq 0
      \end{cases}
    \end{equation*}
    Hence $\chi_E$ is a measurable function. 
  \end{proof}
  \vspace*{.15in}





  \problem \textbf{Carothers 17.8} Suppose that $D = A \cup B$, where $A$ and $B$ are measurable. Show that $f: D \to \RR $ is measurable if and only if $f|_A$ and $f|_B$ are measurable (relative to their respective domains $A$ and $B$, of course).
  \begin{proof} let $D = A \cup B$, where $A$ and $B$ are measurable and suppose that $f: D \to \RR $ is measurable. Without loss of generality consider the function $f|_A$ and note that $\{f|_A \geq \alpha\} = \{f|_A \geq \alpha\} \cap D$ which is measurable, and hence $f|_A$ is measurable. Now suppose $f|_A$ and $f|_B$ are measurable (relative to their respective domains), and consider $f: D \to \RR $. Note that since $D = A \cup B$ we find that $\{f \geq \alpha\} = \{f|_A \geq \alpha\} \cup \{f|_B \geq \alpha\}$ is measurable and therefore $f$ is measurable. 
  \end{proof}
  \vspace*{.15in}




  \problem \textbf{Carothers 17.18} Let $f:[0, 1] \to [0, 1]$ be the Cantor function, and set $g(x) = f(x) + x$. Prove that:
  \begin{enumerate}
    \item[\textbf{a}] $g$ is a homeomorphism of $[0, 1]$ onto $[0, 2]$. In particular, $h = g^{-1}$ is continuous. 
    \begin{proof} First recall that the Cantor function is continuous via Corollary 2.19 and since $g$ is a sum of two continuous functions, $g$ is also continuous. Clearly $g$ is a closed map as $[0, 1]$ is compact and $[0, 2]$ is Hausdorff. What is left to show to prove that $g$ is a homeomorphism we will use the closed map lemma so what is left to show that that $g$ is bijective. Let $x, y \in [0, 1]$ such that $x \neq y$ and note that $g$ is strictly increasing, since $f(x) \geq 0$ and monotone increasing and $x$ is strictly increasing it must follows that $g(x) \neq g(y)$.

      Let $y \in [0, 2]$ and note that since $f$ is continuous and $g(0) = 0$ and $g(1) = 2$ we find by the intermediate value theorem that there exists an $x \in [0, 1]$ such that $g(x) = y$.
    \end{proof}
    


  
    \item[\textbf{b}] $g(\Delta)$ is measurable and $m(g(\Delta)) = 1$. In particular, $g(\Delta)$ contains a nonmeasurable set $A$. 
    \begin{proof} Consider $\Delta^c$ constructed as a countable union of disjoint middle third intervals $I_i$. Since $f$ is a constant $c_i$, on each interval $I_i$ and since $g$ is bijective and each $I_i$ is disjoint in the domain, each $g(I_i) = I_i + c_i$ is disjoint. 
      \begin{equation*}
        g(\Delta^c) = g\left(\bigcup_{i = 1}^\infty I_i\right) = \left(\bigcup_{i = 1}^\infty I_i + c_i\right).
      \end{equation*}
      Clearly this demonstrates that $g(\Delta^c)$ is measurable set, and since $G$ is bijective we find that $g(\Delta^c)^c = g(\Delta)$ is also measurable. It then follows by countable additivity and translation invariance that,  
      \begin{align*}
        m(g(\Delta^c)) &= m\left(\bigcup_{i = 1}^\infty I_i + c_i\right) = \sum_{i = 1}^\infty m(I_i + c_i) = \sum_{i = 1}^\infty m(I_i) = 1 \\
      \end{align*}  
      Since $g$ is bijective we know that $[0, 2] = g(\Delta) \cup g(\Delta^c)$ with $g(\Delta)$ and $g(\Delta^c)$ disjoint we get by additivity that, 
      \begin{equation*}
        m(g(\Delta)) = 2 - m(g(\Delta^c)) = 1. 
      \end{equation*}
      The conclusion that $g(\Delta)$ contains a non-measurable set follows from noting that $m(g(\Delta)) > 0$ and recalling problem 10 from homework 10. 
      
    \end{proof}
    
    

    \item[\textbf{c}] $g$ maps some measurable set onto a nonmeasurable set.
    \begin{proof} As we have shown in the previous part there exists a nonmeasurable set $N \subseteq g(\Delta)$. Since $g$ is bijection, there exists a set $g^{-1}(N) \subseteq \Delta$ and therefore by monotonicity 
      $m^*(g^{-1}(N)) \leq m^*(\Delta) = 0$ so $m^*(g^{-1}(N))$ is a null set and is therefore measurable.
    \end{proof}


    \item[\textbf{d}] $B = g^{-1}(A)$ is Lebesgue measurable but not Borel set. 
    \begin{proof} We have shown that $h = g^{-1}$ is a continuous function, and is therefore measurable. Consider $B = g^{-1}(N)$ from the previous problem. We have shown that $B$ is lebesgue measurable, suppose for the sake of contradiction that $B$ is a Borel set, then it would follow that since $h$ is measurable, it follows that $h^{-1}(B) = g(g^{-1}(N)) = N$ is measurable, which is a contradiction. 
    \end{proof}
    



    \item[\textbf{e}] There is a Lebesgue measurable function $F$ and a continuous function $G$ such that $F \circ G$  is not Lebesgue measurable. 
    \begin{proof} We have show that $g^{-1}$ is a continuous function, and previously we have also shown that indicator functions on measurable sets are also Lebesgue measurable. Now let $G = g^{-1}$ and $F = \chi_B$ note that $F \circ G$ is well defined since $g^{-1}:[0, 2] \to [0, 1]$ and $\chi_B:[0, 1] \to \RR$. Note that for an open set $U = (\frac{1}{2}, \frac{3}{2})$ we get the following, 
      \begin{equation*}
        (F \circ G)^{-1}(U) = G^{-1}(F^{-1}(U)) = g^{-1}(B) = A
      \end{equation*}
      a non-measurable set. 
       \end{proof}
  \end{enumerate}
  \vspace*{.15in}




  \problem \textbf{Carothers 17.32} Check that the conclusion of Theorem 17.8 holds (with the same proof) if 'measurable' is everywhere interpreted as 'Borel measurable'. :Do the same for the four corollaries. What modifications, if any are needed in Corollary 17.12
  \begin{proof} 

  \end{proof}
  \vspace*{.15in}





  \problem \textbf{Carothers 17.33} If $f:(a, b)  \to \RR$ is differentiable, show that $f'$ is Borel measurable. If $f$ is only differentiable a.e, show that $f'$ is still Lebesgue measurable.[Hint: Write $f'$ as the limit of a sequence of continuous functions. ]
  \begin{proof} Suppose $f:(a, b)  \to \RR$ is differentiable and note that by the definition of the derivative, $f'$ can be written as the limit of a sequence of continuous functions
    \begin{equation*}
      f'(x) = \lim_{n \to \infty} \frac{f(x + \frac{b - x}{n}) - f(x)}{\frac{b - x}{n}} = \lim_{n \to \infty} \frac{n}{b - x}\left(f\left(x + \dfrac{b - x}{n}\right) - f(x)\right)
    \end{equation*}
    Note that since $x \in (a, b)$ the pointwise limit of $f'$ is well defined. Note that the function, 
    \begin{equation*}
      f'_n(x) =  \frac{n}{b - x}\left(f\left(x + \dfrac{b - x}{n}\right) - f(x)\right),
    \end{equation*}
    is itself a Borel measurable function and therefore by Corollary 17.12 $f'$ is Borel measurable. 
  \end{proof}


  \vspace*{.15in}





  \problem \textbf{Carothers 17.35} Give an example showing that the requirement that $m(D) < \infty$ cannot be dropped from Egorov's Theorem. 
  \begin{proof} 
  \end{proof}
  \vspace*{.15in}



  % For each $\epsilon > 0$ there exists a measurable subset $E$ of \RR with m(E) < \epsilon such that $f_n$ \to f uniforml on \RR \setminus E. 
  \problem \textbf{Carothers 17.36} If $(f_n)$ converges almost uniformly to $f$, prove that $(f_n)$ converges almost everywhere to $f$. [Hint: For each $k$, choose a set $E_k$  such that $m(E_k )< \frac{1}{k}$ and $f_n \to f$ uniformly off of $E_k$. Then $m(\cap_{k = 1}^\infty E_k) = 0$]. 
  \begin{proof} Let $(f_n)$ be a sequence of functions which converge almost uniformly to $f$. Therefore we can construct a collection of $E_k$ such that $m(E_k) < \frac{1}{k}$ and $f_n \to f$ uniformly on $\RR \setminus E_k$. Let $E = \cap_{k = 1}^\infty E_k $ and note that $E\subseteq E_k$ for all $k$ it follows that $m(E) \leq m(E_k)$ and since this holds for all $k$, it follows that  $m(E) = 0$. Now let $x \in D \setminus E$, so by definition, there exists some $k$ for which $x \in D \setminus E_k$ and since $f_n \to f$ uniformly on $D \setminus E_k$ we know that $f_n(x) \to f(x)$.   
  \end{proof}
  \vspace*{.15in}





  \problem \textbf{Carothers 17.44} Let $f:[a, b] \to [-\infty, \infty]$ be measurable and finite a.e. Prove that there is a sequence of continuous function $(g_n)$ on $[a, b]$ such that $g_n \to f$ a.e. on $[a, b]$. In fact the $g_n$ can be taken to be polynomials
  \begin{proof} Let $f:[a, b] \to [-\infty, \infty]$ be measurable and finite a.e. By Theorem 17.20 we can construct a continuous $(g_n)$ so that $E_n = \{|f - g_n|\geq 2^{ - n - 1}\}$. Now since each $g_n$ are continuous on a closed interval $[a, b]$ we know, by the Weirstrauss Approximation Theorem that there exists a polynomial $p_n$ such that $||g_n - p_n|| \leq 2^{-n - 1}$. Now note that for each $x \in E_n^c$ it follows that by the triangle inequality,
    \begin{equation*}
    |f(x) - p_n(x)| \leq |f - g_n| + |g_n - p_n| < 2^{-n}. 
  \end{equation*}
  So we can redefine $E_n = \{|f - p_n| \geq 2^{-n}\}$ and by Theorem 17.20 we have that $m(E_n) \leq 2^{-n}$.

  Now define ,
  \begin{equation*}
     E = \lim\sup_{n \to \infty} E_n = \bigcap_{n = 1}^\infty\left(\bigcup_{k = n}^\infty E_k\right).
  \end{equation*}
  Note that if $x \in [a, b] \setminus E$ then there exists some $k$ such that such that $x \not\in E_n$ for all $n \geq k$, 
  \begin{equation*}
    |f(x)  - p_n(x)| < 2^{-n}.
  \end{equation*}  

  \end{proof}
  \vspace*{.15in}




  \problem \textbf{Carothers 17.45} Let $f: [a, b] \to \overline{\RR}$ be measurable and finite a.e and let $\epsilon > 0$. Show that there is a continuous function $g$ on $[a, b]$ with $m\{f \neq g\} < \epsilon$.  
  \begin{proof} Let $f: [a, b] \to \overline{\RR}$ be measurable and finite a.e and let $\epsilon > 0$. By the previous problem there exists a sequence of continuous $(g_n)$ on $[a, b]$ such that $g_n \to f$ a.e. on $[a, b]$. By Egorov's Theorem there exists a measurable set $E \subseteq [a, b]$ such that $m(E) < \frac{\epsilon}{2}$ and $(g_n)$ converges uniformly to $f$ on $[a, b] \setminus E$. Now since $E$ is measurable there exists an open $E \subseteq U$ such that $U \setminus E < \frac{\epsilon}{2}$. 
    
    Note that $m([a, b] \setminus F) = m(U) < \epsilon$, and since $E \subseteq U$ we find that $(g_n)$ converges uniformly to $f$ on $[a, b] \setminus U$ as well. Note that $F = [a, b] \setminus U$ is a closed subset of $\RR$ and by Problem 41 there exists a continuous function $g: [a, b] \to \RR$ for which $g(x) = f(x)$ for all $x \in F$ and clearly $m{f \neq g} = m([a, b] \setminus F) = m(U) < \epsilon$.



  \end{proof}
  \vspace*{.15in}








\end{problems}
\end{document}
