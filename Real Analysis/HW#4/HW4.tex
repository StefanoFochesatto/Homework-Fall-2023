\documentclass[12pt]{article}

\usepackage{amssymb,amsmath,amsthm}
\usepackage[top=1in, bottom=1in, left=1.25in, right=1.25in]{geometry}
\usepackage{fancyhdr}
\usepackage{mathtools}
\usepackage{enumerate}
\usepackage{color}

%% Import a package useful for displaying graphics.
\usepackage{graphicx}

% Comment the following line to use TeX's default font of Computer Modern.
\usepackage{times,txfonts}

% Adjust line spacing an indentation
\parindent 0pt
\parskip 6pt plus 1pt

\newtheoremstyle{ex215}% name of the style to be used
  {18pt}% measure of space to leave above the theorem. E.g.: 3pt
  {12pt}% measure of space to leave below the theorem. E.g.: 3pt
  {}% name of font to use in the body of the theorem
  {}% measure of space to indent
  {\bfseries}% name of head font
  {:}% punctuation between head and body
  {2ex}% space after theorem head; " " = normal interword space
  {}% Manually specify head
\theoremstyle{ex215} 

\makeatletter
\newtheorem*{lemmacore}{Lemma \@currentlabel}
\newenvironment{lemma}[1]
{\def\@currentlabel{#1}\lemmacore}
{\endlemmacore}

\newtheorem*{corollarycore}{Corollary \@currentlabel}
\newenvironment{corollary}[1]
{\def\@currentlabel{#1}\corollarycore}
{\endcorollarycore}


%%%%%%%%%%%%%%%%%%%%%%%%%%%%%%%%%%%%%%%%%%%%%%%%%%%%%%%%%%%%%%%%%%%%%%%%
%
% Stuff for getting the name/document date/title across the header
\RequirePackage{fancyhdr}
\pagestyle{fancy}
\fancyfoot[C]{\ifnum \value{page} > 1\relax\thepage\fi}
\fancyhead[L]{\ifx\@doclabel\@empty\else\@doclabel\fi}
\fancyhead[C]{\ifx\@docauthor\@empty\else\@docdate\fi}
\fancyhead[R]{\ifx\@docauthor\@empty\@docdate\else\@docauthor\fi}
\headheight 15pt


\def\doclabel#1{\gdef\@doclabel{#1}}
\doclabel{Use {\tt\textbackslash doclabel\{MY LABEL\}}.}
\def\docdate#1{\gdef\@docdate{#1}}
\docdate{Use {\tt\textbackslash docdate\{MY DATE\}}.}
\def\docauthor#1{\gdef\@docauthor{#1}}
%\docauthor{Use {\tt\textbackslash docauthor\{MY NAME\}}.}
\let\@docauthor\@empty

\newcounter{probcount}
\newcounter{subprobcount}
\newlength\probsep
\newlength\pshrinking
\newif\iffirstprob

% The following commands allow for a handy override of labels in an enumeration,
% without having the user keep track of what level the enumeration is happening at.
% To use them, just put \listlabel{....} as the first command after a \begin{enumerate}.
% The '....' is used to define the label.  You can access the name of the current label
% counter with \thelistlabel.  Hence \listlabel{\textbf{\Alph{\thelistlabel}:}} would
% generate lists with bold labels A:, B:, C:, etc.
\newcommand{\listlabel}[1]{\def\thelistlabel{\@enumctr}%
            \expandafter\def\csname label\@enumctr\endcsname{#1}}


\newenvironment{subproblems}%
  {\begin{enumerate}\listlabel{\alph{\thelistlabel})}%
  %% Make the enumerate environment think we are making a list in a list:
  \global \@newlistfalse}%
  {\end{enumerate}}%

\newenvironment{problems}%
  {\ifhmode\unskip\par\fi\setcounter{probcount}{0}\probsep\parskip
  \sbox\@tempboxa{\textbf{9.}}\pshrinking\wd\@tempboxa\advance\pshrinking\labelsep
  \advance\linewidth -\pshrinking
  \advance\@totalleftmargin\pshrinking
  \advance\leftskip\pshrinking}%
  {\ifhmode\unskip \par\fi\advance\leftskip-\pshrinking}%

\newcommand{\problem}{%
  \setcounter{subprobcount}{0}%
  \stepcounter{probcount}%
  \def\@currentlabel{\arabic{probcount}}%
  \ifhmode
    \unskip \par
  \fi
%  \addpenalty{-4000}%
  \iffirstprob\else\addvspace\probsep\fi
  \firstprobfalse
  \hskip -\labelwidth\hskip -\labelsep 
  \hbox to\labelwidth{\hss\textbf{\arabic{probcount}.}}\hskip\labelsep
}%

%% The localhead command puts a label on top of a paragraph -- handy for labels
%% like "Solution:"
\newskip\restoreparskip
\newcommand{\localhead}[1]{\par\smallskip\textbf{#1}\nobreak\\}%
%% The following arcane code was added to make the localhead enviroment get along
%% with the ams theorem enviroment (which is based on the latex list enviroment).
%% previously, the command was {\par\smallskip\textbf{#1}\\}, but the list
%% enviroment would also add a par at the end of this paragraph, which would result
%% in an extra blank line.  Our solution now is to put in our own par, but with
%% none of the parskip glue that adds extra space between paragraphs.
%%  \restoreparskip\parskip\parskip 0pt\par\nobreak\noindent\parskip\restoreparskip}%
\newcommand\solution{\localhead{Solution:}}
\newcommand\subsol{\stepcounter{subprobcount}\localhead{Solution, part \alph{subprobcount}:}}
\newcommand\subsoleg[1]{\stepcounter{subprobcount}\par\textbf{Solution, part \alph{subprobcount}:} #1\\}
\def\solver#1{(Solution by #1)\\\vskip-12pt}


%% The default 'article' class has captions that label figures Figure 1, etc.  This is too 
%% formal for homework sets, so we change \caption so that it just puts the caption text.
%% Here I have just modified the \@makecaption command from the article class.
%% Maybe I should change this in the future to let authors decide whether or not to label a figure.

\long\def\@makecaption#1#2{%
  \vskip\abovecaptionskip
  \sbox\@tempboxa{#2}%
  \ifdim \wd\@tempboxa >\hsize
    #2\par
  \else
    \global \@minipagefalse
    \hb@xt@\hsize{\hfil\box\@tempboxa\hfil}%
  \fi
  \vskip\belowcaptionskip}

%% The default implementation of the proof environment starts a new paragraph at the start of the proof,
%% with the full parskip spacing. This only looks good following a theorem statement 
%% if \parskip is set to zero.  We've got a medium sized parskip, so we overide this 
%% questionable decision.
\renewenvironment{proof}[1][\proofname]{\par
  \pushQED{\qed}%
  \normalfont \topsep6\p@\@plus6\p@\relax
  \trivlist
  \@topsep \topsep
  \item[\hskip\labelsep
        \itshape
    #1\@addpunct{.}]\ignorespaces
}{%
  \popQED\endtrivlist\@endpefalse
}
\makeatother

% Shortcuts for blackboard bold number sets (reals, integers, etc.)
\newcommand{\Reals}{\ensuremath{\mathbb R}}
\newcommand{\Nats}{\ensuremath{\mathbb N}}
\newcommand{\Ints}{\ensuremath{\mathbb Z}}
\newcommand{\Rats}{\ensuremath{\mathbb Q}}
\newcommand{\Cplx}{\ensuremath{\mathbb C}}
\newcommand{\diam}[1]{\text{diam}(#1)}
\newcommand\converges[1]{\mathrel{\mathop{\longrightarrow}\limits_{#1}}}
\newcommand{\norm}[1]{\left\lVert#1\right\rVert}

%% Some equivalents that some people may prefer.
\let\RR\Reals
\let\NN\Nats
\let\II\Ints
\let\CC\Cplx
\let\ZZ\Ints

\doclabel{Math F641: Homework 4}
\docauthor{Stefano Fochesatto}
\docdate{\today}
\begin{document}
\begin{problems}


%Done
\problem \textbf{Carothers 4.14} Show that the set $A = \{x \in \ell^2: |x_n| \leq 1/n, n = 1, 2, \dots\}$ is a closed set in $\ell_2$ but that $B = \{x \in \ell_2: |x_n|< 1/n, n = 1, 2, \dots\}$ is not an open set.
\begin{proof} Suppose $(x_n) \subseteq A$ with $x_n \to x$. Let $\epsilon > 0$ and note that there exists some $N$ such that if $n \geq N$ we have, 
    \begin{equation*}
   \left(\sum_{k = 1}^{\infty}(x_n(k) - x(k))^2\right)^{\frac{1}{2}} <\epsilon 
  \end{equation*}
  Note that for a fixed $k$ we know, 
  \begin{equation*}
    |x_n(k) - x(k)| \leq \left(\sum_{i = 1}^{\infty}(x_n(i) - x(i))^2\right)^{\frac{1}{2}} < \epsilon.
  \end{equation*}
  Therefore the sequence $(x_n)$ converges term-wise. Recall that by the definition of $A$, for each $k$
  it follows that $|x_n(k)|\leq \frac{1}{k}$ or equivalently $-\frac{1}{k} \leq x_n(k) \leq \frac{1}{k}$. Since $x_n(k) \to x(k)$ it follows that $-\frac{1}{k} \leq x(k) \leq \frac{1}{k}$, and therefore $|x(k)| \leq \frac{1}{k}$. Hence $x \in A$ and $A$ is therefore closed. 
\end{proof}

\begin{proof} First note that the $0$ sequence is clearly in $B$. To show that $B$ is not open, we will show that for all $\epsilon > 0$ there exists an $x \in B_\epsilon(0)$ such that $x \not\in B$.

  Let $\epsilon > 0$ and find $n$ such that $\frac{1}{n} < \epsilon$, then construct the sequence 
  $x$ such that $x(k) = 0$ when $k \neq n$ and $x(k) = \frac{1}{n}$ when $k = n$.
  Note that, 
  \begin{equation*}
    d(0, x) = ||x||_2 = \frac{1}{n} < \epsilon.
  \end{equation*}
  Therefore $x \in B_{\epsilon}(0)$, but clearly $|x(n)| = \frac{1}{n}$ and not less than, so $x \not\in B$. Hence $B$  is not open.  

\end{proof}
\vspace*{.15in}




%Done Maybe clean up supremum argument in the end. 
\problem \textbf{Carothers 4.19} Show that $\diam{A} = \diam{\overline{A}}$. 
\begin{proof} Let $D(A) = \{d(a,b): a, b \in A\}$ and $D(\overline{A}) = \{d(a, b): a, b \in \overline{A}\}$. 

  Let $x \in \overline{D(A)}$. Then by definition there exists a sequence $(d(a, b)_n) \subseteq D(A)$ where $d(a, b)_n \to x$. Choose $(a_n)$ and $(b_n)$
  in $A$ such that $d(a_n, b_n) = d(a, b)_n$. Since $d$ is a continuous function and $d(a, b)_n \to x$ it follows that $a_n \to a$ and $b_n \to b$ where $d(a, b) = x$. By definition $a ,b \in \overline{A}$ and therefore $x \in D(\overline{A})$. Hence $\overline{D(A)} \subseteq D(\overline{A})$. 


  let $x \in D(\overline{A})$. By definition $x = d(a, b)$ where $a, b \in \overline{A}$. Therefore there exists sequences $(a_n)$ and $(b_n)$ in $A$
  such that $a_n \to a$ and $b_n \to b$. Since $d$ is a continuous function, it follows that $d(a_n, b_n) \to d(a, b) = x$, therefore $x \in \overline{D(A)}$. Thus $D(\overline{A}) \subseteq \overline{D(A)}$.

  Since $\overline{D(A)} = D(\overline{A})$ we know that $D(A)$ and $D(\overline{A})$ have the same limit points and therefore $\sup D(A) = \sup D(\overline{A})$ and hence $\diam{A} = \diam{\overline{A}}$.
  

\end{proof}
\vspace*{.15in}






%Done
\problem \textbf{Carothers 5.17} Let $f, g: (M, d) \to (N, \rho)$ be continuous, and let $D$ be a dense subset of $M$. If $f(x) = g(x)$ for all $x \in D$, show that $f(x) = g(x)$ for all $x \in M$. If $f$ is onto, show that $f(D)$ is dense in $N$. 

\begin{proof} Let $x \in M$. Since $D$ is dense, $\overline{D} = M$ so we can find a sequence $(a_n) \subseteq D$ such that $a_n \to x$. Since $f$ and $g$ are continuous functions it follows that $g(a_n) \to g(x)$ and $f(a_n) \to f(x)$. Since $(a_n) \subseteq D$ we know that $g(a_n) = f(a_n)$ and since $N$ is a metric space with unique limits, $f(x) = g(x)$.  
  
  Now suppose $f$ is onto, let $y \in N$ and therefore there exists some $x \in M$ such that $f(x) = y$. Since $D$ is dense in $M$ there exists a sequence $(a_n) \subseteq D$ which converges $a_n \to x$. Since $f$ is continuous it follows that $f(a_n) \to y$. Note $f(a_n) \subseteq f(D)$ and therefore $N \subseteq \overline{f(D)}$. Clearly $\overline{f(D)} \subseteq N$ so hence $N = \overline{f(D)}$. Thus $f(D)$ is dense in $N$. 
\end{proof}





\vspace*{.15in}





\problem \textbf{Carothers 5.19} A function $f: \RR \to \RR$ is said to satisfy a Lipschitz condition if there is a constant $K < \infty$ such that $|f(x) - f(y)| \leq K|x - y|$ for all $x, y \in \RR$. More economically, we may say that $f$ is Lipschitz. Show that $\sin(x)$ is Lipschitz with constant $K = 1$. Prove that a Lipschitz function is (uniformly) continuous. 
\begin{proof} Since $\sin(x)$ is continuous and infinitely differentiable the mean value theorem applies. 
  Let $a, b \in \RR$ such that $a < b$ and by the MVT there exists some $c \in (a, b)$ such that,
  \begin{equation*}
    \cos(c) = \dfrac{\sin(b) - \sin(a)}{b - a}. 
  \end{equation*}
  Taking the absolute value of both sides, we find that 
  \begin{equation*}
    |\cos(c)| = \dfrac{|\sin(b) - \sin(a)|}{|b - a|}.
  \end{equation*}
  Note that $|\cos(c)| \leq 1$ and therefore by substitution we conclude that, 
  \begin{equation*}
    |\sin(b) - \sin(a)| \leq 1|b - a|.
  \end{equation*}
\end{proof}


\begin{proof} Suppose $f: \RR \to \RR$ is Lipschitz with constant $K$. Let $\epsilon > 0$, $x, y \in \RR$ and choose $\delta = \frac{\epsilon}{K}$. Then when $|x - y| < \delta$ it follows that, 
  \begin{equation*}
    |f(x) - f(y)| \leq K|x - y| < K\delta = \epsilon. 
  \end{equation*}   
  Since $\delta$ was chosen solely as a function of $\epsilon$, $f$ is uniformly continuous. 
\end{proof}


\vspace*{.15in}






\problem \textbf{Carothers 5.24} Let $V$ be a normed vector space. If $y \in V$ is fixed, show that the maps $f: \alpha \to \alpha y$ from $\RR$ into $V$, and $g:x \to x + y$, from $V$ into $V$, are continuous. 
\begin{proof} Let $\alpha \in \RR$ and $(\alpha_n) \subseteq \RR$ such that $\alpha_n \to \alpha$. Let $\epsilon > 0$. Then since $\alpha_n \to \alpha$, choose $N$ such that if $n \geq N$, then $|\alpha_n - \alpha| < \frac{\epsilon}{||y||_2}$, it follows that 
  \begin{equation*}
    ||\alpha_ny - \alpha y|| = ||(\alpha_n - \alpha)y|| = |(\alpha_n - \alpha)||\cdot||y|| < \epsilon.
  \end{equation*}
  Therefore $f(\alpha_n) \to f(\alpha)$. 
\end{proof}

\begin{proof} Let $x \in V$ and $(x_n) \subseteq V$ such that $x_n \to x$. Let $\epsilon > 0$. Then since $x_n \to x$, choose $N$ such that $n \geq N$, then $||x_n - x|| < \epsilon$, it follows that
 
  \begin{equation*}
    ||(x_n + y) - (x + y)|| \leq ||x_n - x|| < \epsilon. 
  \end{equation*}
  Therefore $g(x_n) \to g(x)$. 
  
\end{proof}
\vspace*{.15in}






\problem \textbf{Carothers 5.25} A function $f: (M, d) \to (N, \rho)$ is called Lipschitz if there is a constant $k < \infty$ such that $\rho(f(x), f(y)) \leq Kd(x, y)$ for all $x, y \in M$. Prove that a Lipschitz mapping is continuous. 
\begin{proof} Let $x \in M$ and $(x_n) \subseteq M$ such that $x_n \to x$. Let $\epsilon > 0$. Since $x_n \to x$, choose $L$ such that if $n \geq L$, then $d(x_n, x) \leq \epsilon/k$, it follows that, 
\begin{equation*}  
  \rho(f(x_n), f(x)) \leq Kd(x_n, x) < \epsilon. 
\end{equation*}
\end{proof}
\vspace*{.15in}




\problem \textbf{Carothers 5.28} Define $g: \ell^2 \to \RR$ by $g(x) = \sum_{n = 1}^\infty x(k)/k$. Is $g$ continuous. 
\begin{proof} Suppose $x \in \ell^2$ and $(x_n) \subseteq \ell^2$ such that $x_n \to x$. 
  \begin{align*}
    |g(x_n) - g(x)| &= \big|\sum_{k = 1}^\infty x_n(k)\dfrac{1}{k} - \sum_{k = 1}^\infty x(k)\dfrac{1}{k}\big|\\
    &= \big|\sum_{k = 1}^\infty \left(x_n(k) - x(k)\right) \dfrac{1}{k}\big|\\
    &\leq \sum_{k = 1}^\infty |\left(x_n(k) - x(k)\right) \dfrac{1}{k}|\\
  \end{align*}
  By the Cauchy-Schwartz inequality we conclude that, 
  \begin{equation*}
    |g(x_n) - g(x)| \leq \norm{x_n - x}_2 \norm{\dfrac{1}{k}}_2
  \end{equation*}
  Since $x_n \to x$ it follows that $\norm{x_n - x}_2 \to 0$. Note that $\left(\frac{1}{k}\right) \in \ell^2$ so we know that $\norm{\frac{1}{k}}_2 < \infty$. Therefore
  $|g(x_n) - g(x)| \to 0$ and thus $g(x)$ is continuous. 
 

\end{proof}  

\vspace*{.15in}



\problem \textbf{Carothers 5.32} Prove that $f$ is lower semicontinuous if and only if $f(x) \leq \lim \inf_{n \to \infty} f(x_n)$ whenever $x_n \to x$ in $M$. 
\begin{proof} Suppose $f$ is a lower semicontinuous function and $x \in M$ with $(x_n) \subseteq M$ such that $x_n \to x$. Let $m = \lim \inf_{n \to \infty} f(x_n) < \infty$, otherwise concluding $f(x) \leq \lim \inf_{n \to \infty} f(x_n)$ is trivial. By the definition of the limit infimum there are finitely many $n$ such that $m - \epsilon > f(x_n)$ and infinitely many $n$ such that $f(x_n) < m + \epsilon$. Hence it follows that the set $D = \{x \in M : f(x_n)< m+\epsilon\}$ contains infinitely many terms of $x_n$. Consider the subsequence $(x_{n_j})$ of elements of $(x_n)$ which are contained in $D$. Since $f$ is lower semicontinuous, $D$ is closed and since  $x_{n_j} \to x$ it must be the case that $x \in D$. Thus $f(x) \leq \lim \inf_{n \to \infty} f(x_n)$. 
\end{proof}



\begin{proof} Suppose that $f(x) \leq \lim \inf_{n \to \infty} f(x_n)$ whenever $x_n \to x$ in $M$. For the sake of contradiction suppose there exists an $\alpha \in \RR$ such that the set  $D = \{x \in M : f(x_n) \leq \alpha\}$ is not closed. Therefore there exists $(x_n) \subseteq D$ such that $x_n \to x$ with $x \not \in D$. However since $x_n \to x$ in $M$ it follows that $f(x) \leq \lim \inf_{n \to \infty} f(x_n)$. Since $(x_n) \subseteq D$ then we know that $f(x_n) \leq \alpha$ 
  so therefore $\inf_{n \to \infty} f(x_n) \leq \alpha$. Finally we conclude that 
  \begin{equation*}
    f(x) \leq  \lim \inf_{n \to \infty} f(x_n) \leq \alpha.
  \end{equation*}
  Thus $x \in D$, a contradiction. 
\end{proof}



\vspace*{.15in}






\problem \textbf{Carothers 7.5} Prove that $A$ is totally bounded if and only if $\overline{A}$ is totally bounded. 
\begin{proof} Suppose $A$ is totally bounded. Given $\epsilon > 0$, there exists $A_1, \dots, A_n \subset A$, with $\diam{A_i} < \epsilon$ for all $i$, such that $A \subset \cup_{i = 1}^{n}A_i$. Note that by Problem 4.1  
  $\diam{\overline{A_i}} = \diam{A_i} < \epsilon$. 
  
  Let $x \in \overline{A}$. Then there exists $(x_n) \subseteq A$ such that $x_n \to x$. But $A \subseteq \cup_{i = 1}^{n}A_i$, so $(x_n) \subseteq \cup_{i = 1}^{n}A_i$. Since the set $\{A_i\}_{i = 1}^n$ is finite there exists an $A_k$ with a tail of $(x_n)$. Hence there exists $(x_{n_j}) \subset A_k$ such that $x_{n_j} \to x$ and therefore by definition $x \in \overline{A_j}$. Therefore $\overline{A} \subseteq \cup_{i = 1}^n \overline{A_i}$.  

\end{proof}

\begin{proof} Suppose $\overline{A}$ is totally bounded. Given $\epsilon > 0$ there exists $A_1, \dots, A_n \subset \overline{A}$, with $\diam{A_i} < \epsilon$ for all $i$, such that $\overline{A} \subseteq \cup_{i = 1}^nA_i$. 
  Note that $A \subseteq \overline{A} \subseteq \cup_{i = 1}^n A_i$, hence $A$ is totally bounded. 
\end{proof}
\vspace*{.15in}






\problem \textbf{Carothers 7.10} Prove that a totally bounded metric space $M$ is separable. 
\begin{proof} Let $M$ be a totally bounded metric space. Let $\epsilon > 0$ and choose $n$ such that $1/n < \epsilon$. Since $M$ is totally bounded there exists a finite $(1/n)$-net corresponding to balls of radius $1/n$, call it $D_n$. Let $D = \cup_{n = 1}^\infty D_n$, and note that $D$ is a countable union of finite sets, and is therefore countable. What is left to show is that $D$ is dense in $M$. 

  Let $x \in M$ and with $\epsilon > 0$ consider $B_{\epsilon}(x)$. Choose $n$
  such that $1/n < \epsilon$, and note that there exists an element $x' \in D_n$ such that $d(x, x') < 1/n < \epsilon$ and hence $x' \in B_{\epsilon}(x)$. Therefore $D$ is dense in $M$.
\end{proof}
\vspace*{.15in}




\end{problems}
\end{document}
